%%
%% This is file `hustreport-zh-example.tex',
%% generated with the docstrip utility.
%%
%% The original source files were:
%%
%% hustreport.dtx  (with options: `example-zh')
%% 
%% This is a generated file.
%% 
%% Copyright (C) 2013-2014 by Xu Cheng <xucheng@me.com>
%%               2014-2016 by hust-latex <https://github.com/hust-latex>
%% 
%% This work may be distributed and/or modified under the
%% conditions of the LaTeX Project Public License, either version 1.3
%% of this license or (at your option) any later version.
%% The latest version of this license is in
%%   http://www.latex-project.org/lppl.txt
%% and version 1.3 or later is part of all distributions of LaTeX
%% version 2005/12/01 or later.
%% 
%% This work has the LPPL maintenance status `maintained'.
%% 
%% The Current Maintainer of this work is hust-latex Organization.
%% 
%% This work consists of the files hustreport.dtx,
%% hustreport.ins and the derived file hustreport.cls
%% along with its document and example files.
%% 
%% \CharacterTable
%% {Upper-case    \A\B\C\D\E\F\G\H\I\J\K\L\M\N\O\P\Q\R\S\T\U\V\W\X\Y\Z
%%  Lower-case    \a\b\c\d\e\f\g\h\i\j\k\l\m\n\o\p\q\r\s\t\u\v\w\x\y\z
%%  Digits        \0\1\2\3\4\5\6\7\8\9
%%  Exclamation   \!     Double quote  \"     Hash (number) \#
%%  Dollar        \$     Percent       \%     Ampersand     \&
%%  Acute accent  \'     Left paren    \(     Right paren   \)
%%  Asterisk      \*     Plus          \+     Comma         \,
%%  Minus         \-     Point         \.     Solidus       \/
%%  Colon         \:     Semicolon     \;     Less than     \<
%%  Equals        \=     Greater than  \>     Question mark \?
%%  Commercial at \@     Left bracket  \[     Backslash     \\
%%  Right bracket \]     Circumflex    \^     Underscore    \_
%%  Grave accent  \`     Left brace    \{     Vertical bar  \|
%%  Right brace   \}     Tilde         \~}
\documentclass[format=draft,language=chinese,category=academic-report]{hustreport}

\stuno{U201814468}
\title{数据结构实验报告}
\author{王清雨}
\major{计算机科学与技术学院}
\department{计算机卓越工程师}
\advisor{许贵平\hspace{1em}副教授}

\abstract{
    这这是一个\LaTeX{}模板使用实例文件,该模板用于华中科技大学研究生报告写作。

    该模板基于LPPL v1.3发行。

}
\keywords{\LaTeX{},华中科技大学,报告,模板}


\begin{document}

\frontmatter
\maketitle
\makeabstract
\tableofcontents
\listoffigures
\listoftables
\mainmatter

\chapter{基于线性存储结构的线性表实现}\label{chapter:1}

\section{实验目的}\label{sec:1}
通过实验达到
\begin{itemize}
    \item 加深对线性表的概念、基本运算的理解。
    \item 熟练掌握线性表的逻辑结构与物理结构的关系。
    \item 物理结构采用顺序表,熟练掌握线性表的基本运算的实现。
    \item 通过编写完备的测试来保证代码的正确性。
\end{itemize}

\subsection{对线性表对理解}
通过本次实验,我深刻理解了线性表的\textbf{线性}的意义,即所有元素均\textbf{线性}地排列在一起。
\newline
而本次实验的线性表底层是使用顺序表来实现的,在内存上,各个元素也是顺序、线性地排列在一起。
\subsection{对基本运算的理解与实现}
总的来说,线性表的基本运算较简单,主要的难点在插入 (\texttt{insert})元素与删除 (\texttt{delete})元素。
因为需要对线性表的长度 (\texttt{length}),和容量 (\texttt{size})进行改变,
还需要对线性表中的元素 (\texttt{elements})进行移动,
如果操作不当,或者没有对用户的输入进行校验,
可能会产生数组下溢 (\texttt{underflow}),或上溢 (\texttt{overflow}),导致程序出现错误。
\subsection{单元测试}
通过编写完整的单元测试,将可能出现的错误都考虑清楚,尽量实现测试覆盖率达到100\%。
从而保证了程序在正确情况和极端情况下都能正常运行。

% ------------------------------------------------------------

\section{实验内容}\label{sec:2}
\subsection{问题描述}
\subsubsection{线性表的定义}
线性表 (\emph{Linear List})是由$n (n \le 0)$个数据元素(结点)$a[0],a[1],a[2],\dots ,a[n-1]$组成的有限序列。
\newline
其中:
\begin{itemize}
    \item 数据元素可以为任意类型,但同一线性表中元素类型必须相同。
    \item 数据元素的个数$n$定义为表的长度 (\emph{length}),表里没有一个元素时称为空表。
    \item 将非空的线性表 $(n \ge 1)$记作: (a[0],a[1],a[2],\dots ,a[n-1])
数据元素a[i] $(0\le i \le n-1)$只是个抽象符号,其具体含义在不同情况下可以不同
\end{itemize}
\subsection{系统设计}
\subsection{系统实现}
\section{程序设计}\label{sec:3}
\section{实验过程}\label{sec:4}
\subsection{设计线性表类接口}
\subsection{实现线性表基本操作}
\subsection{编写单元测试}
\section{实验结果与分析}\label{sec:5}
\section{心得体会}\label{sec:6}
\subsection{第二层}\label{sec:2}
\subsubsection{第三层}\label{sec:3}
测试测试测试测试测试测试测试测试测试测试测试测试。
\footnote{\label{footnote:1}脚注}

\section{字体}

普通\textbf{粗体}\emph{斜体}

\hei{黑体}\kai{楷体}\fangsong{仿宋}

\section{公式}

单个公式,公式引用:\autoref{eq:1}。
\begin{equation}
 c^2 = a^2 + b^2 \label{eq:1}
\end{equation}

多个公式,公式引用:\autoref{eq:2},\autoref{eq:3}。

\begin{subequations}
\begin{equation}
  F = ma \label{eq:2}
\end{equation}
\begin{equation}
  E = mc^2 \label{eq:3}
\end{equation}
\end{subequations}

\section{罗列环境}

\begin{enumerate}
    \item 第一层\label{item:1}
    \item 第一层
    \begin{enumerate}
        \item 第二层\label{item:2}
        \item 第二层
        \begin{enumerate}
            \item 第三层\label{item:3}
            \item 第三层
        \end{enumerate}
    \end{enumerate}
\end{enumerate}

\begin{description}
    \item[解释环境]  解释内容
\end{description}

\chapter{其他格式测试}

\section{代码环境}

\begin{lstlisting}[language=python]
import os

def main():
    '''
    doc here
    '''
    print 'hello, world' # Abc
    print 'hello, 中文' # 中文
\end{lstlisting}

\section{定律证明环境}

\begin{definition}\label{def:1}
这是一个定义。
\end{definition}
\begin{proposition}\label{proposition:1}
这是一个命题。
\end{proposition}
\begin{axiom}\label{axiom:1}
这是一个公理。
\end{axiom}
\begin{lemma}\label{lemma:1}
这是一个引理。
\end{lemma}
\begin{theorem}\label{theorem:1}
这是一个定理。
\end{theorem}
\begin{proof}\label{proof:1}
这是一个证明。
\end{proof}

\section{算法环境}

\begin{algorithm}[H]
\SetAlgoLined
\KwData{this text}
\KwResult{how to write algorithm with \LaTeX2e }
initialization\;\label{alg_line:1}
\While{not at end of this document}{
read current\;
\eIf{understand}{
go to next section\;
current section becomes this one\;
}{
go back to the beginning of current section\;
}
}
\caption{How to write algorithms}\label{alg:1}
\end{algorithm}

\section{表格}
表格见\autoref{tab:1}。

\begin{table}[!h]
\centering
\caption{一个表格}\label{tab:1}
\begin{tabular}{|c|c|}
\hline
a & b \\
\hline
c & d \\
\hline
\end{tabular}
\end{table}
\section{图片}
图片见\autoref{fig:1}。图片格式支持eps,png,pdf等。多个图片见\autoref{fig:2},分开引用:\autoref{fig:2-1},\autoref{fig:2-2}。

\begin{figure}[!h]
\centering
\includegraphics[width=.4\textwidth]{fig-example.pdf}
\caption{一个图片}\label{fig:1}
\end{figure}

\begin{figure}[!h]
\centering
  \begin{subfigure}[b]{0.3\textwidth}
  \includegraphics[width=\textwidth]{fig-example.pdf}
  \caption{图片1}\label{fig:2-1}
  \end{subfigure}
  ~
  \begin{subfigure}[b]{0.3\textwidth}
  \includegraphics[width=\textwidth]{fig-example.pdf}
  \caption{图片2}\label{fig:2-2}
  \end{subfigure}
\caption{多个图片}\label{fig:2}
\end{figure}

\section{参考文献示例}
这是一篇中文参考文献\cite{TEXGURU99};这是一篇英文参考文献\cite{knuth};同时引用\cite{TEXGURU99,knuth}。

\section[\textbackslash{}autoref 测试]{\texttt{\textbackslash{}autoref} 测试}

\begin{description}
  \item[公式] \autoref{eq:1}
  \item[脚注] \autoref{footnote:1}
  \item[项] \autoref{item:1},\autoref{item:2},\autoref{item:3}
  \item[图] \autoref{fig:1}
  \item[表] \autoref{tab:1}
  \item[附录] \autoref{appendix:1}
  \item[章] \autoref{chapter:1}
  \item[小节] \autoref{sec:1},\autoref{sec:2},\autoref{sec:3}
  \item[算法] \autoref{alg:1},\autoref{alg_line:1}
  \item[证明环境] \autoref{def:1},\autoref{proposition:1},\autoref{axiom:1},\autoref{lemma:1},\autoref{theorem:1},\autoref{proof:1}
\end{description}

\backmatter

\begin{ack}
致谢正文。
\end{ack}

\bibliography{ref-example}

\appendix

\begin{publications}
    \item 论文1
    \item 论文2
\end{publications}

\chapter{这是一个附录}\label{appendix:1}
附录正文。


\end{document}
\endinput
%%
%% End of file `hustreport-zh-example.tex'.
