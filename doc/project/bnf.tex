\chapter{系统详细设计}

% 指作者对自己的研究工作的详细表述。
% 要求论理正确、论据确凿、逻辑性强、层次分明、表达确切

\section{语法的巴克斯范式定义}\label{sec:bnf}

下面根据巴克斯范式定义编译器系统所支持的C语言子集
\footnote{使用EBNF能够更简单地对语言进行描述,
	但纯粹的BNF格式对编写代码有比较好的帮助作用,因此这里使用纯粹的BNF格式}。
任务书中要求的全部特性均满足,并且在其基础上增加了:
\begin{itemize}
	\item 各种单目运算符(包括取地址运算符)
	\item 完全的指针类型(不限制级别)
	\item 枚举类型(\mintinline{cpp}|enum|)
	\item 各种位运算符
	\item 移位运算符
	\item 类型转换
	\item 三目运算符
\end{itemize}

与语法分析相同,我们采用自顶向下的定义方式。

\subsection{程序的基本结构}

对于一个C语言程序源文件来说,它其实只由两部分组成:{\bf 声明}(declaration)
和{\bf 函数定义}(function-definition),
因此其巴克斯范式定义如\autoref{bnf:program}所示。

\begin{bnf}
	\bnfprod{program}{\bnfpn{declaration}\bnfsp\bnfts{;}\bnfor\bnfpn{function-definition}}\label{bnf:program}
\end{bnf}

\subsection{声明与函数定义的巴斯克范式定义}

\paragraph{声明}
声明包括了函数的声明和外部变量的声明,其定义如\autoref{bnf:declaration}至\autoref{bnf:type}所示。
% declaration
\begin{bnf}
	\bnfprod{declaration}
	{\bnfpn{type-spec}\bnfsp\bnfpn{variable-declaration-list}}\label{bnf:declaration}\\
	\bnfmore{\bnfor\bnfpn{type-spec}\bnfsp\bnfpn{id}\bnfsp\bnfts{(}\bnfsp\bnfpn{param-list}\bnfsp\bnfts{)}}\\
	\bnfmore{\bnfor\bnfpn{enum-spec}}\\
	% ------------------ %
	\bnfprod{variable-declaration-list}
	{\bnfpn{id}}\\
	\bnfmore{\bnfor\bnfpn{variable-declaration-list}\bnfsp\bnfts{,}\bnfsp\bnfpn{id}}\\
	% ------------------ %
	\bnfprod{param-list}
	{\bnfpn{param-list}\bnfsp\bnfts{,}\bnfsp\bnfpn{param}}\\
	\bnfmore{\bnfor\bnfpn{param}}\\
	% ------------------ %
	\bnfprod{param}
	{\bnfpn{type}\bnfsp\bnfpn{id}}\\
	% ------------------ %
	\bnfprod{type-spec}
	{\bnfpn{type}\bnfor\bnfts{void}}\\
	\bnfprod{type}
	{\bnfts{char}\bnfor\bnfts{int}}\\
	\bnfmore{\bnfor\bnfpn{type}\bnfsp\bnfts{*}}\label{bnf:type}
\end{bnf}

\paragraph{函数定义}
函数定义的定义则如\autoref{bnf:func}所示,其函数体由大括号包含的若干条语句组成
(当然可以为空\footnote{$\lambda$表示空})。
% function-definition
\begin{bnf}
	\bnfprod{function-definition}
	{\bnfpn{type-spec}\bnfsp\bnfpn{id}\bnfsp\bnfts{(}\bnfsp\bnfpn{param-list}\bnfsp\bnfts{)}\bnfsp\bnfts{\{}\bnfsp\bnfpn{statement-list}\bnfsp\bnfts{\}}}\label{bnf:func}\\
	\bnfprod{statement-list}
	{\bnfes\bnfor\bnfpn{statement}\bnfor\bnfpn{statement-list}\bnfsp\bnfpn{statement}}
\end{bnf}

\subsection{枚举类型的巴斯克范式定义}

\paragraph{枚举类型}
枚举类型的定义如下所示
% enum related
\begin{bnf}
	\bnfprod{enum-spec}
	{\bnfts{enum} \bnfsp \bnfpn{id}\bnfsp \bnfts{\{} \bnfsp \bnfpn{enumerator-list} \bnfsp \bnfts{\}}}\\
	\bnfmore{\bnfor \bnfts{enum} \bnfsp\bnfts{\{}\bnfsp\bnfpn{enumerator-list}\bnfsp\bnfts{\}}}\\
	\bnfmore{\bnfor \bnfts{enum} \bnfsp \bnfpn{id}}\\
	% ------------------ %
	\bnfprod{enumerator-list}
	{\bnfpn{enumerator}}\\
	\bnfmore{\bnfor\bnfpn{enumerator-list}\bnfsp\bnfts{,}\bnfsp\bnfpn{enumerator}}\\
	% ------------------ %
	\bnfprod{enumerator}
	{\bnfpn{id}}\\
	\bnfmore{\bnfor \bnfpn{id}\bnfsp\bnfts{=}\bnfsp\bnfpn{const-expression}}
\end{bnf}

\subsection{语句和表达式的巴斯克范式定义}

\paragraph{语句}
语句是C语言较为基本的语言模块,作为过程型语言,其一条条语句执行的特点影响了后面的很多语言。

本编译器系统支持的语句有:
\begin{itemize}
	\item \mintinline{cpp}|if|语句
	\item \mintinline{cpp}|while|语句
	\item \mintinline{cpp}|for|语句
	\item 赋值语句
	\item 局部变量声明语句
	\item \mintinline{cpp}|return|语句
	\item 表达式语句
	\item 复合语句
	\item 空语句
\end{itemize}
这里给出的顺序与下面的定义分别对应。
% statement
\begin{bnf}
	%------------------------%
	\bnfprod{statement}
	{\bnfpn{if-statement}}\\
	\bnfmore{\bnfor\bnfpn{while-statement}}\\
	\bnfmore{\bnfor\bnfpn{for-statement}}\\
	\bnfmore{\bnfor\bnfpn{assign-statement}}\\
	\bnfmore{\bnfor\bnfpn{local-decl-statement}}\\
	\bnfmore{\bnfor\bnfpn{return-statement}}\\
	\bnfmore{\bnfor\bnfpn{expr-statement}}\\
	\bnfmore{\bnfor\bnfts{\{}\bnfsp\bnfpn{statement}\bnfsp\bnfts{\}}}\\
	\bnfmore{\bnfor\bnfts{;}}
\end{bnf}

% all-kinds of statement
\begin{bnf*}
	\bnfprod{if-statement}
	{\bnfts{if}\bnfsp\bnfts{(}\bnfsp\bnfpn{expression}\bnfsp\bnfts{)}\bnfsp\bnfpn{statement}}\\
	\bnfmore{\bnfor\bnfts{if}\bnfsp\bnfts{(}\bnfsp\bnfpn{expression}\bnfsp\bnfts{)}\bnfsp\bnfpn{statement}\bnfsp\bnfts{else}\bnfsp\bnfpn{if-statement}}\\
	\bnfmore{\bnfor\bnfts{if}\bnfsp\bnfts{(}\bnfsp\bnfpn{expression}\bnfsp\bnfts{)}\bnfsp\bnfpn{statement}\bnfsp\bnfts{else}\bnfsp\bnfpn{statement}}\\
	\bnfprod{while-statement}
	{\bnfts{while}\bnfsp\bnfts{(}\bnfsp\bnfpn{expression}\bnfsp\bnfts{)}\bnfsp\bnfpn{statement}}\\
	\bnfprod{for-statement}
	{\bnfts{for}\bnfsp\bnfts{(}\bnfsp\bnfpn{expression}\bnfsp\bnfts{;}\bnfsp\bnfpn{expression}\bnfsp\bnfts{;}\bnfsp\bnfpn{expression}\bnfsp\bnfts{)}\bnfsp\bnfpn{statement}}\\
	\bnfprod{assign-statement}{\bnfpn{id}\bnfsp\bnfts{=}\bnfsp\bnfpn{expression}\bnfsp\bnfts{;}}\\
	\bnfprod{local-decl-statement}{\bnfpn{type}\bnfsp\bnfpn{id}\bnfsp\bnfts{=}\bnfsp\bnfpn{expression}\bnfsp\bnfts{;}}\\
	\bnfprod{return-statement}
	{\bnfts{return}\bnfsp\bnfpn{expression}\bnfsp\bnfts{;}}\\
	\bnfmore{\bnfor\bnfts{return}\bnfsp\bnfts{;}}\\
	\bnfprod{expr-statement}{\bnfpn{expression}\bnfsp\bnfts{;}}
\end{bnf*}

\paragraph{表达式}
表达式是组成语句的基本模块,其组合方式十分多样,可由标量(scalar)、标识符(id),
以及它们的组合和它们和各种运算符(operator)的组合构成。
本编译器系统支持:
\begin{itemize}
	\item 双目运算符构成的语句
	\item 单目运算符构成的语句
	\item 三目运算符语句(问号语句)
	\item 由括号扩起来的语句
	\item 类型转换语句
	\item 标识符语句
	\item 标量语句
\end{itemize}
这里给出的顺序与下面的定义分别对应。

% expression
\begin{bnf}
	\bnfprod{expression}
	{\bnfpn{expression}\bnfsp\bnfpn{binary-opertor}\bnfsp\bnfpn{expression}}\\
	\bnfmore{\bnfor\bnfpn{unary-opertor}\bnfsp\bnfpn{expression}}\\
	\bnfmore{\bnfor\bnfpn{expression}\bnfsp\bnfts{?}\bnfsp\bnfpn{expression}\bnfsp\bnfts{:}\bnfsp\bnfpn{expression}}\\
	\bnfmore{\bnfor\bnfts{(}\bnfsp\bnfpn{expression}\bnfsp\bnfts{)}}\\
	\bnfmore{\bnfor\bnfts{(}\bnfsp\bnfpn{expression}\bnfsp\bnfts{)}\bnfsp\bnfpn{expression}}\\
	\bnfmore{\bnfor\bnfpn{id}}\\
	\bnfmore{\bnfor\bnfpn{scalar}}
\end{bnf}

其中,双目运算符包括
\begin{itemize}
	\item 算数运算符
	\item 比较运算符
	\item 逻辑运算符
	\item 位运算符
\end{itemize}

其定义如下:
% binary operator
\begin{bnf*}
	\bnfprod{binary-operator}
	{\bnfts{+}}\\
	\bnfmore{\bnfor\bnfts{-}}\\
	\bnfmore{\bnfor\bnfts{*}}\\
	\bnfmore{\bnfor\bnfts{/}}\\
	\bnfmore{\bnfor\bnfts{\%}}\\
	\bnfmore{\bnfor\bnfts{<}}\\
	\bnfmore{\bnfor\bnfts{>}}\\
	\bnfmore{\bnfor\bnfts{<=}}\\
	\bnfmore{\bnfor\bnfts{>=}}\\
	\bnfmore{\bnfor\bnfts{!=}}\\
	\bnfmore{\bnfor\bnfts{==}}\\
	\bnfmore{\bnfor\bnfts{\&\&}}\\
	\bnfmore{\bnfor\bnfts{||}}\\
	\bnfmore{\bnfor\bnfts{<<}}\\
	\bnfmore{\bnfor\bnfts{>>}}\\
	\bnfmore{\bnfor\bnfts{\&}}\\
	\bnfmore{\bnfor\bnfts{|}}\\
	\bnfmore{\bnfor\bnfts{\^{}}}
\end{bnf*}

单目运算符定义如下:
% unary operator
\begin{bnf*}
	\bnfprod{unary-opertor}
	{\bnfts{-}}\\
	\bnfmore{\bnfor\bnfts{+}}\\
	\bnfmore{\bnfor\bnfts{!}}\\
	\bnfmore{\bnfor\bnfts{\~{}}}\\
	\bnfmore{\bnfor\bnfts{++}}\\
	\bnfmore{\bnfor\bnfts{--}}\\
	\bnfmore{\bnfor\bnfts{*}}\\
	\bnfmore{\bnfor\bnfts{\&}}
\end{bnf*}

\subsection{其他定义}

在上面的定义过程中,用到了其他的定义如下:
\begin{bnf}
	\bnfprod{type}
	{\bnfts{char}\bnfor\bnfts{int}}\\
	\bnfmore{\bnfor\bnfpn{type}\bnfsp\bnfts{*}}\\
	\bnfprod{id}
	{\bnfpn{letter}}\\
	\bnfmore{\bnfor\bnfpn{id}\bnfsp\bnfpn{letter}}\\
	\bnfmore{\bnfor\bnfpn{id}\bnfsp\bnfpn{digit}}\\
	\bnfmore{\bnfor\bnfpn{id}\bnfsp\bnfts{\_}}\\
	\bnfprod{letter}
	{\bnfts{a}\bnfsk\bnfts{z}\bnfor\bnfts{A}\bnfsk\bnfts{Z}}\\
	\bnfprod{digit}
	{\bnfts{0}\bnfsk\bnfts{9}}\\
	\bnfprod{scalar}
	{\bnfpn{number}\bnfor\bnfpn{string}}\\
	\bnfprod{number}
	{\bnftd{decimal number}\bnfor\bnftd{octal number}\bnfor\bnftd{hexital number}}\\
	\bnfprod{string}
	{\bnfts{"}\bnfsp\bnftd{0 or more letter}\bnfsp\bnfts{"}}
\end{bnf}

至此,本编译器支持的C语言子集就定义完成了。可以看出,我们这个定义还是比较完善的
一个C语言子集,绝大多数功能都能够实现(如\mintinline{cpp}|struct|可通过指针{\tt +offset}的方式实现)。
% end
