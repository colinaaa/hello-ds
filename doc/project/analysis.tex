\chapter{系统实现与测试}

\section{系统实现}

\subsection{开发环境}

主要的开发环境如下:

\begin{itemize}
	\item 系统环境:MacOS Catalina 10.15.3 Darwin Kernel Version 19.3.0
	\item 编译器:Apple clang version 11.0.0 (clang-1100.0.33.17)
	\item 自动构建工具:cmake version 3.16.5
	\item 编辑器:NVIM v0.4.3
	\item 标准库版本:libc++ 11.0.0
\end{itemize}

此外,系统还有部分开发在Linux下完成,开发环境如下:

\begin{itemize}
	\item 系统环境:GNU/Linux Arch Linux-5.5.9.arch1-2
	\item 编译器:g++ (Arch Linux 9.3.0-1) 9.3.0
	\item 自动构建工具:cmake version 3.16.5
	\item 编辑器:NVIM v0.4.3
	\item 标准库版本:glibc 2.31-2
\end{itemize}

由于手边没有Windows系统电脑,不确定是否能在Windows下编译运行。

\subsection{交互实现}

系统采用命令行交互,与大多现代编译器相同,支持诸多命令行参数。
默认情况下,如果源代码没有问题,则自动退出,无任何输出。

若指定{\tt -t}选项,则会输出源程序的抽象语法树。
其中,{\tt -i}选项可以指定抽象语法树每一层缩进的大小。

若指定{\tt -o}选项,则会将源代码输出至指定文件。

\paragraph{{\tt main}函数}

首先解析命令行参数,并将选项通过\autoref{lst:cmdinterface}中的接口传入编译器。

\begin{listing}[htb]
\begin{minted}{cpp}
void setSrc(const bool v = true);
void setDebug(const bool v = true);
void setTree(const bool v = true);
void setIndent(const int v);
void setOutputName(const std::string &v);
void setOutput(const bool v = true);
\end{minted}
\caption{命令行选项接口}\label{lst:cmdinterface}
\end{listing}

接着,通过{\tt run}函数指定要编译的源文件,并开始编译。

\mint{cpp}|int run (const std::string &filename);|

\subsection{词法分析器实现}

词法分析器主要由一个函数\mintinline{cpp}|void next()|实现。


{\tt next()}读取对象的{\tt tk, p, pp}等变量的值,并按照\autoref{alg:FSM}
进行词法分析,判定当前标记对应的类型,并将对应的值存储在变量{\tt tk}中。

\begin{listing}[hbt]
\begin{minted}{cpp}
void next() {
  if (tk == '\n') {
    // newline
  } else if (tk == '#') {
    // deal with marco
  } else if (tk >= '0' && tk <= '9') {
    // deal with number
  }
  // ...
  else {
    // unknown token
  }
}
\end{minted}
\caption{词法分析器实现}\label{lst:leximpl}
\end{listing}

\autoref{lst:leximpl}中省略了其中繁琐的分支判断和每条分支中自动状态机的状态转换,
详细代码见\autoref{ch:src}。

\subsection{语法分析器实现}

语法分析器主要由两个函数构成,\mintinline{cpp}|void expr(int level)|和
\mintinline{cpp}|void stmt()|,分别对表达式和语句进行分析。

\paragraph{语句分析}

\begin{listing}[hbt]
\begin{minted}{cpp}
void stmt() {
  if (tk == If) {
    // deal with if statement
    expr(Assign);
    // ...
    stmt();
  } else if (tk == While) {
    // deal with while statement
    expr(Assign);
    // ...
    stmt();
  }
  // ...
  else {
    // unknown statement
    // throw error
  }
}
  \end{minted}
  \caption{语法分析器-语句分析}
\end{listing}

在语句分析中,会递归调用表达式分析和语句分析来进行分析,于此同时,也会将
收集到的信息放入抽象语法树中,供后续使用。

由于不同的语句的结构很不同,因此需要手动编写每条语句递归程度。
例如处理\mintinline{cpp}|for|语句的分支中需要调用三次{\tt expr()}进行表达式分析
\footnote{判断括号中的三个语句是否正确},
需要调用四次{\tt next()}进行词法分析\footnote{判断括号和分号是否正确},
需要调用一次{\tt stmt()}进行语句分析。

\paragraph{表达式分析}

在表达式分析中,会递归调用表达式分析和词法分析来进行分析,于此同时,也会将
收集到的信息放入抽象语法树中,供后续使用。

\begin{listing}[hbt]
  \begin{minted}{cpp}
void expr(int lev) {
  if (tk == Num) {
    // deal with number
    next();
    // ...
  } else if (tk == Id) {
    // deal with identifier
    next();
    // ...
  }
  // ...
  else {
    // unknown expression
  }
}
  \end{minted}
  \caption{语法分析器-表达式分析}
\end{listing}

与语句分析一样,由于不同的表达式形式不同,也需要对不同的表达式进行单独分析。

% end
