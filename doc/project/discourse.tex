\section{有关数据结构的定义}

\subsection{符号表定义}

首先,编译器需要建立一个符号表,记录所有定义
过的全局变量和函数,以确保不产生重名。

符号表的定义我采用了线性表的方式,较为简单。
每个符号由一个哈希函数被散列到一个整数值,
顺序存储在线性表中。因为一个源文件中符号并不会很多,
因此使用顺序表的性能不会太差,而比链表节省了
许多空间,实现也更加简单。\autoref{lst:symbol}中为其定义。

\begin{listing}[hbt]
	\mint{cpp}|int *sym;|
	\caption{符号表定义}\label{lst:symbol}
\end{listing}

\subsection{数据段定义}

编译器需要建立一个数据段,记录所有的常量,
静态变量等,与符号表类似,同样用线性表的方式实现。
\autoref{lst:databss}中为其定义。

\begin{listing}[hbt]
	\mint{cpp}|int *data;|
	\caption{数据段定义}\label{lst:databss}
\end{listing}

\subsection{词法分析器定义}

词法分析器中的数据结构比较简单,如\autoref{lst:lexdef}所示。
包含一个指针指向当前的字符,一个指针指向当前的行,
一个指针指向当前标识符在符号表中的位置,
一个整型值代表当前标记的种类,
一个整型代表当前标记的值,
一个整型代表当前标记的类型。

\begin{listing}[hbt]
	\begin{minted}{cpp}
char *p;
char *lp;
int *id;
int tk;
int ival;
int ty;
\end{minted}
	\caption{词法分析器定义}\label{lst:lexdef}
\end{listing}

\subsection{语法分析器定义}

语法分析器需要频繁地调用词法分析器,并获得一些值,
因此词法分析器中的数据语法分析器也可以访问,\autoref{lst:syntaxdef}为其定义。

此外,语法分析器还需要生成抽象语法树,
这主要通过一个栈来实现。
每当遇到一个语素,分析得到它的类型及有关
信息后,将其压入栈中。

\begin{listing}[hbt]
	\mint{cpp}|int *n;|
	\caption{语法分析器定义}\label{lst:syntaxdef}
\end{listing}

这里没有使用结构体的原因如下:

\begin{itemize}
	\item 首先,我想实现编译器的自举,即自己能够编译
	      自己。而因为语言定义中没有给出结构体的定义,所以
	      这里就没有使用结构体。
	\item 其次,使用栈的方式大大简化了代码。因为抽象
	      语法树每个节点的孩子个数不定,因此需要使用孩子-兄弟
	      表示法。然而这样会使得编写代码较为麻烦,且使用了很多
	      不必要的空间。
	\item 此外,我的栈中并非单纯存储数据,其中也存储了
	      一些指针,这也有一部分链表和树的思想。但最主要
	      还是通过栈的方式进行存储。
\end{itemize}

\subsection{抽象语法树打印器}

由于抽象语法树由栈来实现,因此打印器也通过栈来实现。
不同于抽象语法树,打印器的栈只需要存储其
空白格的长度,行为较简单,无需自定义,
因此使用了C++中的{\tt stack}容器。
\autoref{lst:astdef}中为其定义。

\begin{listing}[hbt]
\mint{cpp}|std::stack<int> s;|
\caption{抽象语法树定义}\label{lst:astdef}
\end{listing}

打印器还具有以下方法:

\begin{listing}[hbt]
\mint{cpp}|void log(const std::string&, int);|
\caption{{\tt log}方法}
\end{listing}

{\tt log}方法,打印一个节点的信息,根据第二个参数调整
缩进的大小。

\begin{listing}[hbt]
\mint{cpp}|int top();|
\caption{{\tt top}方法}
\end{listing}

{\tt top}方法,如果栈中有元素则弹出该元素并返回其值,
否则返回默认的缩进大小。

\subsection{其他定义}\label{subsec:other_definition}

除了上述模块的定义外,编译器还通过枚举类型定义了一些符号,
主要用于关键字、运算符和类型。

\begin{listing}[hbt]
\begin{minted}{cpp}
enum Token {  };
enum Operator {  };
enum Type { CHAR, INT, PTR };
\end{minted}
\caption{枚举类型定义}
\end{listing}

对于指针类型,最初是用统一的枚举类型{\tt PTR}来代表所有类型的指针,
例如
\mint{cpp}|id[Type] = PTR;|
就代表{\tt id}的类型为指针类型,但究竟其指向的是整型还是另一个指针就不得而知。
但后来我用一种方法对其进行了优化,详见\autoref{sec:optim}

\section{主要算法设计}\label{sec:algorithm}

\subsection{有限自动状态机}

有限自动状态机的算法较为简单,其基本思想如\autoref{fig:lex_FSM}。
主要复杂的部分在于定义其转移状态,而这一步在定义语言的过程中已经基本
完成了。

算法的主要结构如\autoref{alg:FSM}所示:

\begin{algorithm}[hbt]
	\caption{有限自动状态机算法}\label{alg:FSM}
	\SetAlgoLined{}
	\KwData{current token}
	\KwResult{identify this token}
	\SetKw{EOF}{EOF}
	\SetKw{Return}{return}

	\While{not encounter \EOF}{%
		\uIf{token is number}{%
			deal with number\;
			\Return\;
		}\uElseIf{token is string}{%
			deal with string\;
			\Return\;
		}\uElseIf{token is identifier}{%
			deal with identifier\;
			\Return\;
		}\ElseIf{token is operator}{%
			deal with operator\;
			\Return\;
		}
	}
\end{algorithm}

\subsection{递归下降}

语法分析主要采用递归下降法进行分析。根据\autoref{sec:bnf}
所定义的语法,可以发现如果从左向右来分析的话,
可以完全确定对应的标记的类型,而不需要进行推测\cite{kernighan2006c},
这样的语言被称为{\it LL (k) 语言}。

例如\autoref{lst:llk}:

\begin{listing}[hbt]
\mint{cpp}|for (i = 0; i < 5; ++i) {}|
\caption{LL(k)语言示例}\label{lst:llk}
\end{listing}

它的分析结果如\autoref{fig:for}

\begin{figure}[hbt]
	\centering
	\begin{forest}
		[for,draw
		[{=}[i][0]]
		[<[i][5]]
		[++[i]]
		[{\{\}}]
		]
	\end{forest}
	\caption{{\tt for}语句}\label{fig:for}
\end{figure}

在我们读到{\tt for}的时候,我们就能够确定这是一条
for语句,其接下来的结构也很清楚,因此我们就可以
使用自顶向下的递归下降的方式来进行语法分析\cite{aho1986compilers,kernighan2006c},
而不需要进行回溯分析。
这种分析器被称为{\it LL(k) Parser}。

算法的主要结构如\autoref{alg:recurive_down}所示:

\begin{algorithm}[hbt]
	\caption{递归下降算法}\label{alg:recurive_down}
	\KwIn{string $\omega$}
	\KwOut{left-most derivation of $\omega$ or error}
	\BlankLine{}
	\Begin{%
		\Repeat{$\omega$ match some pattern}{%
			read more of $\omega$\;
		}
		\uIf{match some statement}{%
			get the pattern of statement\;
			recursive go down following the pattern\;
		}\uElseIf{match another statement}{%
			recursive go down\;
		}\Else{%
			error\;
		}
	}
\end{algorithm}

就这样,我们通过自顶向下构建抽象语法树,最终将全部声明与定义
组合在一个抽象语法森林中(或者可以给所有根节点一个共同的父亲)。
如\autoref{fig:AST_forset}所示。

\begin{figure}[hbt]
	\centering
	\begin{forest}
		[
		{some definition},draw[
		\dots[
			{some stmt}, draw
				[\dots[{some expr}, draw][{some expr}, draw]]
		][
			{some stmt}, draw
				[\dots[{some expr}, draw][{some expr}, draw]]
		][
			{some stmt}, draw
				[\dots[{some expr}, draw][{some expr}, draw]]
		]]]
	\end{forest}

	\caption{抽象语法森林}\label{fig:AST_forset}
\end{figure}

\section{系统优化}\label{sec:optim}

在完成了要求的基础功能后,我还对编译器进行了一定程度的优化,
使得它更加接近成熟的现代编译器。

\subsection{常量优化}

在源代码中,我们常常会看到像\autoref{lst:constexpr}这样的表达式:

\begin{listing}[hbt]
\mint{cpp}|1 << 10, !(~(0xff & -1))|
\caption{常量表达式}\label{lst:constexpr}
\end{listing}

其中所有的操作数都是常数,因此这个表达式的值是可以在编译期间被完全确定下来的。
因此,我们完全没有必要将整个表达式都放在抽象语法树中,而是可以直接计算出结果,
并用结果作为一个标量表达式来替换原表达式的结果%
\footnote{这里说的替换并不是先建立抽象语法树再进行优化、替换,而是在建立语法树的同时就做了替换}。
这也是现代编译器普遍具有的一个常见优化\cite{appel2004modern,aho1986compilers,Advanced-compiler-optim}。

\subsection{语义分析优化}\label{subsec:semantic}

如同\autoref{subsec:analysis}所述,经过语法分析后,编译器已经完全了解源代码所要
表达的含义,然而这仍然是不足的。就针对于C语言来说,你的语法正确了,并不代表你的
程序一定是正确的。例如\autoref{lst:semanticerror}

\begin{listing}[hbt]
\begin{minted}{cpp}
void foo() {
  return 0; // error!
}
\end{minted}
\caption{语义错误示例}\label{lst:semanticerror}
\end{listing}

上面这个{\tt foo}函数单单从语法上来说,完全符合定义,没有问题。
然而,它的返回类型为\mintinline{cpp}|void|,本不应该返回任何值,
但它的函数体中却返回了一个整型{\tt 0}。这就出现了典型的语义错误 (semantic error)。

再比如\autoref{lst:semanticerror2}这个例子:
\begin{listing}[hbt]
\begin{minted}{cpp}
void foo() {
  return;
}
void bar() {
  int x;
  x = foo(); // error!
}
\end{minted}
\caption{语义错误示例}\label{lst:semanticerror2}
\end{listing}

在{\tt bar()}函数的函数体内,一个对整型变量的赋值发生了明显的错误,因为{\tt foo()}
函数的返回值为\mintinline{cpp}|void|。然而在语法上,上面这个例子完全正确。这也再一次体现了,
只有词法分析和语法分析是无法保证源代码的正确性的。

以上给出的例子都是会使代码完全崩溃的例子,当然也有诸如隐式类型转换这种C语言标准
允许,但可能会导致程序出现错误的问题存在。这些都是属于语义分析所能够解决的问题。
研究表面,目前现代编译器对于词法和语法的分析基本都能做到很完善,差距就体现在后端生成代码
的质量和前端语义分析的智能程度。作为一名编译器使用者,这也就体现在编译器是否能够
帮助你尽可能地降低发生错误的概率。

例如这段代码:
\begin{listing}[hbt]
\begin{minted}{cpp}
int sum, a, b;
sum = a + b;
\end{minted}
\caption{语义分析示例}
\end{listing}

经过语义分析后,编译器可以得到以下信息:

\begin{figure}[hbt]
	\centering
	\begin{forest}
		[
			{=}
				[{number\\[-2ex]\scriptsize sum},align=center]
				[
					{+}
						[{number\\[-2ex]\scriptsize a},align=center]
						[{number\\[-2ex]\scriptsize b},align=center]
				]
		]
	\end{forest}
	\caption{语义分析}
\end{figure}

就可以根据这些信息判断这个语句不仅仅在语法上正确,并且在语义上也正确。

当然,我的语义分析还不够完善,不能够保证能分析出全部的语义错误,但是对于简单的
错误还是能够分析出来,并给出相应的提示或警告。

\subsection{类型优化}

C语言作为一个静态类型语言,每一个变量都有其相对应的元素可谓是其一大特点。
尽管编译为目标代码后并不存在类型的问题,但如果具有类型,可以使\autoref{subsec:semantic}
所述的语义分析更加准确,能够在编译器规避更多可能发生的错误。

在刚刚完成系统时,如\autoref{subsec:other_definition}所述,我的编译器对指针的支持
仅仅停留在general pointer的层面。也就是说所有指针是同一个类型,不论它指向的是
字符型还是整型变量或者是另一个指针变量。这确实符合C语言标准,因为所有指针在字节
层面上确实是完全相同的类型\cite{kernighan2006c}。然而,如果能够知道一个指针所指向
变量的类型,就能够作出更精确的语义分析,能够帮助程序员规避可能发生的错误。

于是,我对类型进行了一些优化。对于一个指向基本类型(不包括指针的类型)的指针,
它的类型值就为该基本类型对应的值加上{\tt PTR}。而对于一个指向指针的指针,其类型值
就再加上{\tt PTR}。而\mintinline{cpp}|void|型的指针的类型值则直接为{\tt PTR}。

举例来说:

\begin{listing}[hbt]
\begin{minted}{cpp}
enum Type { CHAR, INT, PTR };
int x;        // id[Type] == INT
int *ptr;     // id[Type] == INT + PTR
int **pptr;   // id[Type] == INT + PTR + PTR
void *rptr;   // id[Type] == PTR
\end{minted}
\caption{类型优化}
\end{listing}

经此优化,编译器便支持了带有类型的指针,同时也支持了空指针和多重指针。因此能
在语义分析中做的更准确,更好地提供辅助作用。
% end
