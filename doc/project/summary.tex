\chapter{总结与展望}

\section{课程设计总结}

\subsection{工作总结}

整个课程设计从寒假开始,共花费一周时间进行设计,之后花费一周时间完成第一版开发,
再后来的时间进行测试和报告编写,直至\zhtoday{}完成课程设计。

整个项目共有源代码近3000行,报告的源代码有近7000行。详细信息见\autoref{lst:src_count}。

\begin{listing}[hbt]
	\scriptsize
\begin{verbatim}
github.com/AlDanial/cloc v 1.84  T=0.14 s (866.9 files/s, 86345.4 lines/s)
-------------------------------------------------------------------------------
Language                     files          blank        comment           code
-------------------------------------------------------------------------------
TeX                             31            399            160           6766
C++                             42            290            242           2756
C/C++ Header                     9            104             23            452
JSON                             5              0              0            255
Markdown                         3             56              0            158
CMake                           18             46              7            129
YAML                             5              9              6             84
C                                6             13              5             77
make                             1             18              1             50
TypeScript                       1              4              0             39
Bourne Shell                     1              1              0              2
-------------------------------------------------------------------------------
SUM:                           122            940            444          10768
-------------------------------------------------------------------------------
\end{verbatim}
\caption{源代码行数}\label{lst:src_count}
\end{listing}

\subsection{系统的优点}
本次实验中,我成功地完成了任务要求,并且在要求的语法基础上做了许多扩展,
支持了指针类型等C语言常用语法。同时,代码组织结构化,模块分工明确,并且绝大部分
代码都含有注释。

同时,系统还在许多方面做了优化,提高了效率与生成树的质量。

此外,系统很多想法都借鉴与相关操作系统概念,如符号表,数据段等,运行栈等,
可以说和操作系统紧密相连。

\subsection{系统的不足}

系统在语义分析方面还不够完善,并且报错信息也有时不够准确。此外,在设计语法时,
特地选取了适合使用递归下降分析的语法,使得语法解析变得比较简单。
并且系统只支持C语言的一个子集,并不能说是一个完整的C语言编译器前端。

相比于成熟的现代化编译器,我的系统可以说连玩具的算不上,不仅仅在功能上不足,
还在实用性上差距很大,并且运行效率也不高。

\section{工作展望}

完成项目后,我深深地感受到了不足。在今后的学习中,我将进一步探索编译原理有关知识,
尤其是更加理论化的概念与证明,并且将其与操作系统相结合,争取能有一天能够使用
自己创造的语言写一个能编译自身的编译器,并且用它们来创造一个自己的操作系统,
最后把操作系统运行在自己设计的CPU上。

\chapter{体会与感想}

我从听说有编译原理这门课以来,就一直对其充满了好奇。在看到任务的时候,我几乎是
立即决定选择这个任务,实现一个简单的编译器前端。当然,由于编译原理是一个理论性
非常强的分支,因此在初期遇到了巨大的困难。龙书\cite{aho1986compilers}、
虎书\cite{appel2004modern}、鲸书\cite{muchnick1997advanced}我都翻看了,但确实
不能很快理解通透。再后来看了网上但一些例子,自己也试着动手尝试,慢慢地才完成了
整个项目。

从整个项目的完成过程中,我不仅仅在实践中体会编译器及其内部数据结构的运行,
更锻炼了编程的能力和解决问题的能力。

之所以采用了很多操作系统相关概念,是因为我在假期中学习了深入理解计算机系统\cite{bryant2003computer},
其中的很多操作系统思想令我印象深刻。同时我也对把C语言代码变成汇编语言的过程产生
了及其浓厚的兴趣,因为它看起来是如此的繁琐,几乎是不可能完成的任务。

当然,在整个过程中我也意识到了自己的知识的不足,以后也会更刻苦的学习,争取能够
做出自己的编译器。
