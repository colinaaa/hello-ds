%!TEX program = lualatex
\documentclass[format=draft,language=chinese,category=academic-report]{hustreport}

\stuno{U201814468}
\title{数据结构实验报告}
\author{王清雨}
\major{计算机科学与技术学院}
\department{计算机卓越工程师}
\advisor{许贵平\hspace{1em}副教授}

\abstract{
    这这是一个\LaTeX{}模板使用实例文件,该模板用于华中科技大学研究生报告写作。

    该模板基于LPPL v1.3发行。

}
\keywords{\LaTeX{},华中科技大学,报告,模板}


\begin{document}

\frontmatter
\maketitle
\makeabstract
\tableofcontents
\listoffigures
\listoftables
\mainmatter

\chapter{基于线性存储结构的线性表实现}\label{chapter:1}

\section{实验目的}\label{sec:1}
通过实验达到
\begin{itemize}
    \item 加深对线性表的概念、基本运算的理解。
    \item 熟练掌握线性表的逻辑结构与物理结构的关系。
    \item 物理结构采用顺序表,熟练掌握线性表的基本运算的实现。
    \item 通过编写完备的测试来保证代码的正确性。
\end{itemize}

\subsection{对线性表对理解}
通过本次实验,我深刻理解了线性表的\textbf{线性}的意义,即所有元素均\textbf{线性}地排列在一起。
\newline
而本次实验的线性表底层是使用顺序表来实现的,在内存上,各个元素也是顺序、线性地排列在一起。
\subsection{对基本运算的理解与实现}
总的来说,线性表的基本运算较简单,主要的难点在插入 (\texttt{insert})元素与删除 (\texttt{delete})元素。
因为需要对线性表的长度 (\texttt{length}),和容量 (\texttt{size})进行改变,
还需要对线性表中的元素 (\texttt{elements})进行移动,
如果操作不当,或者没有对用户的输入进行校验,
可能会产生数组下溢 (\texttt{underflow}),或上溢 (\texttt{overflow}),导致程序出现错误。
\subsection{单元测试}
通过编写完整的单元测试,将可能出现的错误都考虑清楚,尽量实现测试覆盖率达到100\%。
从而保证了程序在正确情况和极端情况下都能正常运行。

% ------------------------------------------------------------

\section{实验内容}\label{sec:2}
\subsection{问题描述}
\subsubsection{线性表的定义}
\begin{definition}\label{def:linear list}
    线性表 (\emph{Linear List})是由$n (n \le 0)$个数据元素(结点)$a[0],a[1],a[2],\dots ,a[n-1]$组成的有限序列。
\end{definition}
其中:
\begin{itemize}
    \item 数据元素可以为任意类型,但同一线性表中元素类型必须相同。
    \item 数据元素的个数$n$定义为表的长度 (\emph{length}),表里没有一个元素时称为空表。
    \item 将非空的线性表 $(n \ge 1)$记作: (a[0],a[1],a[2],\dots ,a[n-1])。\footnote{数据元素a[i] $(0\le i \le n-1)$只是个抽象符号,其具体含义在不同情况下可以不同。}
    \item 对于非空的线性表,每一个数据元素都有其确定的位置,
        例如$a_{0}$是第一个数据元素,$a_{n-1}$是最后一个数据元素,$a_i$是第i+1个数据元素。
    \item 而对于每一个数据元素,除了首元素和尾元素外,均有前驱和后继。
\end{itemize}
\subsubsection{顺序表的定义}
在本次实验中,采用\emph{顺序表}的形式来存储数据。
\begin{definition}\label{def:list}
    是指用一组\emph{地址连续}的存储单元\emph{依次}存储数据元素的线性结构。
\end{definition}
通过本次实验,我理解到了顺序表的精髓所在:
通过数据元素\emph{物理}存储的相邻关系来反映数据元素之间\emph{逻辑}上的\emph{相邻关系}。
\newline
顺序表的存储特点是:只要确定了\emph{起始位置},表中任一元素的地址都通过下列公式得到:
$location(a_i) = location(a_1) + (i-1) * size  1\le i\le n$ 其中,$size$是元素占用存储单元的长度。
因此,顺序表可以方便地进行随机存取元素,在数据的存取上时间复杂度为$O(1)$,但于此同时,
线性表在进行元素的插入 (\texttt{insert}),和数据的删除 (\texttt{delete})时需要移动元素,因此会有$O(n)$的复杂度。
\subsubsection{实验需要完成的基本操作}
\begin{enumerate}
\item 初始化表:\texttt{InitaList(L)},\newline \textbf{\emph{初始条件}}是线性表L不存在已存在 。\newline \textbf{\emph{操作结果}}是构造一个空的线性表。
\item 销毁表:\texttt{DestroyList(L)},\newline \textbf{\emph{初始条件}}是线性表L已存在 。\newline \textbf{\emph{操作结果}}是销毁线性表L。
\item 清空表:\texttt{ClearList(L)},\newline \textbf{\emph{初始条件}}是线性表L已存在 。\newline \textbf{\emph{操作结果}}是将L重置为空表。
\item 判定空表:\texttt{ListEmpty(L)},\newline \textbf{\emph{初始条件}}是线性表L已存在 。\newline \textbf{\emph{操作结果}}是若L为空表则返回\textbf{TRUE},否则返回\textbf{FALSE}。
\item 求表长:\texttt{ListLength(L)},\newline \textbf{\emph{初始条件}}是线性表已存在 。\newline \textbf{\emph{操作结果}}是返回L中数据元素的个数。
\item 获得元素:\texttt{GetElem(L,i,e)},\newline \textbf{\emph{初始条件}}是线性表已存在,1\le i\le ListLength(L) 。\newline \textbf{\emph{操作结果}}是用e返回L中第i个数据元素的值。
\item 查找元素:\texttt{LocateElem(L,e,compare())},\newline \textbf{\emph{初始条件}}是线性表已存在 。\newline
    \textbf{\emph{操作结果}}是返回L中第1个与e满足关系\texttt{compare()}关系的数据元素的位序,若这样的数据元素不存在,则返回值为0。
\item 获得前驱:\texttt{PriorElem(L,cur,pre)},\newline \textbf{\emph{初始条件}}是线性表L已存在 。\newline \textbf{\emph{操作结果}}是若cur是L的数据元素,且不是第一个,则用pre返回它的前驱,否则操作失败,pre无定义。
\item 获得后继:\texttt{NextElem(L,cur,next)},\newline \textbf{\emph{初始条件}}是线性表L已存在 。\newline \textbf{\emph{操作结果}}是若cur是L的数据元素,且不是最后一个,则用next返回它的后继,否则操作失败,next无定义。
\item 插入元素:\texttt{ListInsert(L,i,e)},\newline \textbf{\emph{初始条件}}是线性表L已存在且非空,1\le i \le ListLength(L)+1 。\newline \textbf{\emph{操作结果}}是在L的第i个位置之前插入新的数据元素e。
\item 删除元素:\texttt{ListDelete(L,i,e)},\newline \textbf{\emph{初始条件}}是线性表L已存在且非空,1\le i\le ListLength(L) 。\newline \textbf{\emph{操作结果}}:删除L的第i个数据元素,用e返回其值。
\item 遍历表:\texttt{ListTraverse(L,visit())},\newline \textbf{\emph{初始条件}}是线性表L已存在,\newline \textbf{\emph{操作结果}}是依次对L的每个数据元素调用函数visit()。
\end{enumerate}
\subsection{系统设计}
\subsubsection{总体设计}
本系统采用\emph{顺序表}作为线性表的物理结构,实现线性表的基本运算。
\par
系统开始运行的时候默认不使用文件中的数据,但是用户随时可以将文件中的数据导入到内存中,同时提供数据保存的功能。
\subsubsection{有关常量和类型定义}
采取\texttt{C++}中的模版类来使线性表支持所有类型的数据,
而底层采用\emph{数组}来存储数据元素,即\texttt{\_elem}成员,
为防止手动管理内存而造成内存泄露的问题,
采用\texttt{unique\_ptr}\footnote{需要C++11及以上的编译器支持}对底层的指针进行管理。
\par
此外,线性表类中还有两个成员变量,\texttt{\_length}和\texttt{\_size},
分别代表当前线性表的已有元素数量与能够存储的元素的数量。
\par
另外,作为封装,将\texttt{\_length, \_size, \_elem}都声明为私有成员,防止被非友元函数篡改。
\begin{lstlisting}[language=c++]
#include <string>
#include <memory>
namespace Lab1 {
template <typename T>
class List {
  using File = std::string;
 private:
  std::size_t _length;  // len 已有元素数量
  std::size_t _size;    // cap 能够存储的元素数量
  std::unique_ptr<T[]> _elem;
  // ...
}
}
\end{lstlisting}
对于程序中可能出现的错误,进行统一规定:
\begin{enumerate}
    \item 对于用户输入不正确导致的数组上溢,统一抛出 (\texttt{throw}) \texttt{std::overflow\_error}。
    \item 对于用户输入不正确导致的数组下溢,统一抛出 (\texttt{throw}) \texttt{std::underflow\_error}。
    \item 对于其他可能发生的错误,统一抛出 (\texttt{throw}) \texttt{std::runtime\_error}。
\end{enumerate}
\subsubsection{函数设计}
\begin{lstlisting}[language=c++]
namespace Lab1 {
template <typename T>
class List {
  // ...
 public:
  List();
  explicit List(std::size_t);      // init with size
  List(std::initializer_list<T>);  // init with initializer_list
  // begin() and end() implement range-based loop
  auto begin() const -> T * { return _length > 0 ? &_elem[0] : nullptr; }
  //! There is an array overflow that I used for the range-base for loop
  //! should implement Iterator class instead of this
  auto end() const -> T * { return _length > 0 ? &_elem[_length] : nullptr; }
  inline auto size() const { return _size; }
  inline auto length() const { return _length; }
  auto empty() -> bool; // test whether list is empty
  auto operator[](std::size_t) noexcept(false) -> T &; // a more common way to get and set elements
  auto get(std::size_t) -> T &; // get() gets an element
  auto locate(T, std::function<bool(const T &, const T &)> &&) -> std::size_t; // locate() finds an element
  auto prior(const T &) -> T &; // prior finds the prior element
  auto next(const T &) -> T &; // next finds the next element
  auto traverse(std::function<void(T &)> &&) -> void;
  auto resize(std::size_t) -> void;
  auto insert(std::size_t, const T &) -> void; // insert an element after the index
  auto insert(const T &) -> void; // insert an element to the tail
  auto remove(std::size_t, T &) -> void; // remove an element from index, returning by param
  auto remove(std::size_t) -> T; // remove an element from index returning by return value
  auto save(File &&f) -> void; // save to file
  auto load(File &&f) -> void; // load from file
}
}
\end{lstlisting}
\subsubsection{算法设计}
由于大部分基本操作都较为简单,因此在这里就不一一列举。
只给出插入 (\texttt{insert})和删除 (\texttt{delete})元素的算法设计。
\newline
\begin{algorithm}[H]
    \SetAlgoLined
    \KwIn{elem, index}
    \KwOut{none}
    \If{index out of range}{ throw error}
    \If{length == size}{ resize the list }
    \For{i=length; i\ge{index}; i++}{
        move the i th element backword
    }
    length++;
    \\
    elements[i]=elem
\caption{Insert}\label{alg:insert}
\end{algorithm}
\begin{algorithm}
    \SetAlgoLined
    \KwIn{index}
    \KwOut{out}
    \If{index out of range}{ throw error }
    out=elements[index]
    \\
    \For{i=index; i < length; i++}{
        move the i th element forward
    }
    length--
\caption{Delete}\label{alg:delete}
\end{algorithm}
\subsection{系统实现}
\subsubsection{开发环境}
本次实验中使用的环境如下:
\begin{enumerate}
    \item 操作系统版本 Darwin X86\_64 Kernel Version 18.7.0
    \item 编译器及其版本 clang++ version 10.0.1 (Apple LLVM version 10.0.1)
    \item 自动编译工具 CMake version 3.15.4
    \item 编程环境 NeoVim
\end{enumerate}
同时,本次实验的一部分代码是在另一环境下编写并测试通过:
\begin{enumerate}
    \item 操作系统及其版本:Arch Linux 5.3.7.arch1-2 (x86\_64)
    \item 编译器及其版本 clang++ version 9.0.0
    \item 自动编译工具 CMake version 3.15.4
    \item 编程环境 Visual Studio Code
\end{enumerate}
\subsubsection{源代码}
源代码见\autoref{appendix:lab1},测试代码见\autoref{appendix:test1}
\subsubsection{测试}
本系统使用了\textit{Catch2}\footnote{url: https://github.com/catchorg/Catch2}作为测试框架,
对所有源代码编写了\textbf{完善}的单元测试,可以做到所有边界情况和越界情况以及正常情况全部覆盖,
同时程序中的每一个分支也都进行测试,使得测试覆盖度基本达到100\%。
\section{实验结果与分析}\label{sec:4}
\section{心得体会}\label{sec:5}

\begin{table}[!h]
\centering
\caption{一个表格}\label{tab:1}
\begin{tabular}{|c|c|}
\hline
a & b \\
\hline
c & d \\
\hline
\end{tabular}
\end{table}

\section{参考文献示例}
这是一篇中文参考文献\cite{TEXGURU99} ; 这是一篇英文参考文献\cite{knuth} ; 同时引用\cite{TEXGURU99,knuth}。

\backmatter

\begin{ack}
致谢正文。
\end{ack}

\bibliography{ref-example}

\appendix

\chapter{实验源代码}\label{appendix:1}
\section{文件结构}
\lstinputlisting{../tree.dat}
\section{实验一源代码}\label{appendix:lab1}
\subsection{线性表实现代码}
\emph{构造函数与初始化——List.cc}
\lstinputlisting[language=c++]{../src/lab/1/List.cc}
\emph{元素操作——elem.cc}
\lstinputlisting[language=c++]{../src/lab/1/elem.cc}
\emph{插入删除等操作——modify.cc}
\lstinputlisting[language=c++]{../src/lab/1/modify.cc}
\emph{文件操作——fs.cc}
\lstinputlisting[language=c++]{../src/lab/1/fs.cc}
\emph{CMake配置}
\lstinputlisting{../src/lab/1/CMakeLists.txt}
\subsection{测试代码}\label{appendix:test1}
\emph{构造函数与初始化测试——List\_test.cc}
\lstinputlisting[language=c++]{../test/lab/1/List_test.cc}
\emph{元素操作测试——elem\_test.cc}
\lstinputlisting[language=c++]{../test/lab/1/elem_test.cc}
\emph{插入删除等操作测试——modify\_test.cc}
\lstinputlisting[language=c++]{../test/lab/1/modify_test.cc}
\emph{文件操作测试——fs\_test.cc}
\lstinputlisting[language=c++]{../test/lab/1/fs_test.cc}

\end{document}
\endinput
%%
%% End of file `hustreport-zh-example.tex'.
