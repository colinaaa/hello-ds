%!TEX program = lualatex
\documentclass[format=draft,language=chinese,category=academic-report]{hustreport}
\usepackage{booktabs}
\usepackage[section]{placeins}
%\usepackage [english]{babel}
%\usepackage [autostyle, english = american]{csquotes}
%\MakeOuterQuote{"}
\stuno{U201814468}
\title{数据结构实验}
\author{王清雨}
\major{计卓1801}
\department{计算机科学与技术学院}
\advisor{许贵平\hspace{1em}副教授}

\keywords{\LaTeX{},华中科技大学,报告,模板}


\begin{document}

\frontmatter
\maketitle
%\makeabstract
\tableofcontents
%\listoffigures
%\listoftables

\mainmatter
\chapter{基于线性存储结构的线性表实现}\label{chapter:1}

\section{实验目的}\label{sec:1}
通过实验达到
\begin{itemize}
    \item 加深对线性表的概念、基本运算的理解。
    \item 熟练掌握线性表的逻辑结构与物理结构的关系。
    \item 物理结构采用顺序表,熟练掌握线性表的基本运算的实现。
    \item 通过编写完备的测试来保证代码的正确性。
\end{itemize}

\subsection{对线性表对理解}
通过本次实验,我深刻理解了线性表的\textbf{线性}的意义,即所有元素均\textbf{线性}地排列在一起。
\newline
而本次实验的线性表底层是使用顺序表来实现的,在内存上,各个元素也是顺序、线性地排列在一起。
\subsection{对基本运算的理解与实现}
总的来说,线性表的基本运算较简单,主要的难点在插入 (\texttt{insert})元素与删除 (\texttt{delete})元素。
因为需要对线性表的长度 (\texttt{length}),和容量 (\texttt{size})进行改变,
还需要对线性表中的元素 (\texttt{elements})进行移动,
如果操作不当,或者没有对用户的输入进行校验,
可能会产生数组下溢 (\texttt{underflow}),或上溢 (\texttt{overflow}),导致程序出现错误。
\subsection{单元测试}
通过编写完整的单元测试,将可能出现的错误都考虑清楚,尽量实现测试覆盖率达到100\%。
从而保证了程序在正确情况和极端情况下都能正常运行。

% ------------------------------------------------------------

\section{实验内容}\label{sec:2}
    实验内容主要分为一下三个部分:
\begin{enumerate}
    \item 问题描述
    \item 系统设计
    \item 系统实现
\end{enumerate}
\section{问题描述}
\subsection{线性表的定义}
\begin{definition}\label{def:linear list}
    线性表 (\emph{Linear List})是由$n (n \le 0)$个数据元素(结点)$a[0],a[1],a[2],\dots ,a[n-1]$组成的有限序列。
\end{definition}
其中:
\begin{itemize}
    \item 数据元素可以为任意类型,但同一线性表中元素类型必须相同。
    \item 数据元素的个数$n$定义为表的长度 (\emph{length}),表里没有一个元素时称为空表。
    \item 将非空的线性表 $(n \ge 1)$记作: (a[0],a[1],a[2],\dots ,a[n-1])。\footnote{数据元素a[i] $(0\le i \le n-1)$只是个抽象符号,其具体含义在不同情况下可以不同。}
    \item 对于非空的线性表,每一个数据元素都有其确定的位置,
        例如$a_{0}$是第一个数据元素,$a_{n-1}$是最后一个数据元素,$a_i$是第i+1个数据元素。
    \item 而对于每一个数据元素,除了首元素和尾元素外,均有前驱和后继。
\end{itemize}
\subsection{顺序表的定义}
在本次实验中,采用\emph{顺序表}的形式来存储数据。
\begin{definition}\label{def:list}
    是指用一组\emph{地址连续}的存储单元\emph{依次}存储数据元素的线性结构。
\end{definition}
通过本次实验,我理解到了顺序表的精髓所在:
通过数据元素\emph{物理}存储的相邻关系来反映数据元素之间\emph{逻辑}上的\emph{相邻关系}。
\newline
顺序表的存储特点是:只要确定了\emph{起始位置},表中任一元素的地址都通过下列公式得到:
$location(a_i) = location(a_1) + (i-1) * size  1\le i\le n$ 其中,$size$是元素占用存储单元的长度。
因此,顺序表可以方便地进行随机存取元素,在数据的存取上时间复杂度为$O(1)$,但于此同时,
线性表在进行元素的插入 (\texttt{insert}),和数据的删除 (\texttt{delete})时需要移动元素,因此会有$O(n)$的复杂度。
\subsection{实验需要完成的基本操作}
\begin{enumerate}
\item 初始化表:\texttt{InitList(L)},\newline \textbf{\emph{初始条件}}是线性表L不存在已存在 。\newline \textbf{\emph{操作结果}}是构造一个空的线性表。
\item 销毁表:\texttt{DestroyList(L)},\newline \textbf{\emph{初始条件}}是线性表L已存在 。\newline \textbf{\emph{操作结果}}是销毁线性表L。
\item 清空表:\texttt{ClearList(L)},\newline \textbf{\emph{初始条件}}是线性表L已存在 。\newline \textbf{\emph{操作结果}}是将L重置为空表。
\item 判定空表:\texttt{ListEmpty(L)},\newline \textbf{\emph{初始条件}}是线性表L已存在 。\newline \textbf{\emph{操作结果}}是若L为空表则返回\textbf{TRUE},否则返回\textbf{FALSE}。
\item 求表长:\texttt{ListLength(L)},\newline \textbf{\emph{初始条件}}是线性表已存在 。\newline \textbf{\emph{操作结果}}是返回L中数据元素的个数。
\item 获得元素:\texttt{GetElem(L,i,e)},\newline \textbf{\emph{初始条件}}是线性表已存在,1\le i\le ListLength(L) 。\newline \textbf{\emph{操作结果}}是用e返回L中第i个数据元素的值。
\item 查找元素:\texttt{LocateElem(L,e,compare())},\newline \textbf{\emph{初始条件}}是线性表已存在 。\newline
    \textbf{\emph{操作结果}}是返回L中第1个与e满足关系\texttt{compare()}关系的数据元素的位序,若这样的数据元素不存在,则返回值为0。
\item 获得前驱:\texttt{PriorElem(L,cur,pre)},\newline \textbf{\emph{初始条件}}是线性表L已存在 。\newline \textbf{\emph{操作结果}}是若cur是L的数据元素,且不是第一个,则用pre返回它的前驱,否则操作失败,pre无定义。
\item 获得后继:\texttt{NextElem(L,cur,next)},\newline \textbf{\emph{初始条件}}是线性表L已存在 。\newline \textbf{\emph{操作结果}}是若cur是L的数据元素,且不是最后一个,则用next返回它的后继,否则操作失败,next无定义。
\item 插入元素:\texttt{ListInsert(L,i,e)},\newline \textbf{\emph{初始条件}}是线性表L已存在且非空,1\le i \le ListLength(L)+1 。\newline \textbf{\emph{操作结果}}是在L的第i个位置之前插入新的数据元素e。
\item 删除元素:\texttt{ListDelete(L,i,e)},\newline \textbf{\emph{初始条件}}是线性表L已存在且非空,1\le i\le ListLength(L) 。\newline \textbf{\emph{操作结果}}:删除L的第i个数据元素,用e返回其值。
\item 遍历表:\texttt{ListTraverse(L,visit())},\newline \textbf{\emph{初始条件}}是线性表L已存在,\newline \textbf{\emph{操作结果}}是依次对L的每个数据元素调用函数visit()。
\end{enumerate}
\section{系统设计}
\subsection{总体设计}
本系统采用\emph{顺序表}作为线性表的物理结构,实现线性表的基本运算。
\par
系统开始运行的时候默认不使用文件中的数据,但是用户随时可以将文件中的数据导入到内存中,同时提供数据保存的功能。
\subsection{有关常量和类型定义}
采取\texttt{C++}中的模版类来使线性表支持所有类型的数据,
而底层采用\emph{数组}来存储数据元素,即\texttt{\_elem}成员,
为防止手动管理内存而造成内存泄露的问题,
采用\texttt{unique\_ptr}\footnote{需要C++11及以上的编译器支持}对底层的指针进行管理。
\par
此外,线性表类中还有两个成员变量,\texttt{\_length}和\texttt{\_size},
分别代表当前线性表的已有元素数量与能够存储的元素的数量。
\par
另外,作为封装,将\texttt{\_length, \_size, \_elem}都声明为私有成员,防止被非友元函数篡改。
\begin{lstlisting}[language=c++]
#include <string>
#include <memory>
namespace Lab1 {
template <typename T>
class List {
  using File = std::string;
 private:
  std::size_t _length;  // len 已有元素数量
  std::size_t _size;    // cap 能够存储的元素数量
  std::unique_ptr<T[]> _elem;
  // ...
}
}
\end{lstlisting}
对于程序中可能出现的错误,进行统一规定:
\begin{enumerate}
    \item 对于用户输入不正确导致的数组上溢,统一抛出 (\texttt{throw}) \texttt{std::overflow\_error}。
    \item 对于用户输入不正确导致的数组下溢,统一抛出 (\texttt{throw}) \texttt{std::underflow\_error}。
    \item 对于其他可能发生的错误,统一抛出 (\texttt{throw}) \texttt{std::runtime\_error}。
\end{enumerate}
\subsection{函数设计}
头文件中的函数原型声明见下面的代码段
\begin{lstlisting}[language=c++,float,floatplacement=h]
namespace Lab1 {
template <typename T>
class List {
  // ...
 public:
  List();
  explicit List(std::size_t);      // init with size
  List(std::initializer_list<T>);  // init with initializer_list
  // begin() and end() implement range-based loop
  auto begin() const -> T * { return _length > 0 ? &_elem[0] : nullptr; }
  //! There is an array overflow that I used for the range-base for loop
  //! should implement Iterator class instead of this
  auto end() const -> T * { return _length > 0 ? &_elem[_length] : nullptr; }
  inline auto size() const { return _size; }
  inline auto length() const { return _length; }
  auto empty() -> bool; // test whether list is empty
  auto operator[](std::size_t) noexcept(false) -> T &; // a more common way to get and set elements
  auto get(std::size_t) -> T &; // get() gets an element
  auto locate(T, std::function<bool(const T &, const T &)> &&) -> std::size_t; // locate() finds an element
  auto prior(const T &) -> T &; // prior finds the prior element
  auto next(const T &) -> T &; // next finds the next element
  auto traverse(std::function<void(T &)> &&) -> void;
  auto resize(std::size_t) -> void;
  auto insert(std::size_t, const T &) -> void; // insert an element after the index
  auto insert(const T &) -> void; // insert an element to the tail
  auto remove(std::size_t, T &) -> void; // remove an element from index, returning by param
  auto remove(std::size_t) -> T; // remove an element from index returning by return value
  auto save(File &&f) -> void; // save to file
  auto load(File &&f) -> void; // load from file
}
}
\end{lstlisting}\label{code:header1}
\par
函数时间复杂度分析见\autoref{tab:timeandspace1}
\begin{table}[h]
\centering
\caption{函数时间与空间复杂度分析}
\label{tab:timeandspace1}
\begin{tabular}{@{}ccccc@{}}
\toprule
编号                          & 名称  & 函数签名 & 时间复杂度 & 空间复杂度 \\ \toprule
    \multicolumn{1}{c|}{1}  & 初始化表 & \texttt{ InitList()} & $O(1)$ &  $O(1)$ \\
    \multicolumn{1}{c|}{2}  & 销毁表& \texttt{ DestroyList()} & $O(1)$ &  $O(1)$ \\
    \multicolumn{1}{c|}{3}  & 清空表& \texttt{ ClearList()} & $O(1)$ &  $O(1)$   \\
    \multicolumn{1}{c|}{4}  & 判定空表& \texttt{ empty()} & $O(1)$ &  $O(1)$     \\
    \multicolumn{1}{c|}{5}  & 求表长 & \texttt{ length()} & $O(1)$ &  $O(1)$     \\
    \multicolumn{1}{c|}{6}  & 获得元素 & \texttt{ get()} & $O(1)$ &  $O(1)$      \\
    \multicolumn{1}{c|}{7}  & 查找元素 & \texttt{ locate()} & $O(n)$ &  $O(1)$   \\
    \multicolumn{1}{c|}{8}  & 获得前驱 & \texttt{ prior()} & $O(1)$ &  $O(1)$    \\
    \multicolumn{1}{c|}{9}  & 获得后继 & \texttt{ next()} & $O(1)$ &  $O(1)$     \\
    \multicolumn{1}{c|}{10}  & 插入元素 & \texttt{ insert()} & $O(n)$ &  $O(1)$  \\
    \multicolumn{1}{c|}{11}  & 删除元素 & \texttt{ delete()} & $O(n)$ &  $O(1)$  \\
    \multicolumn{1}{c|}{12}  & 遍历表 & \texttt{ traverse()}   & $O(n)$ & $O(1)$ \\ \bottomrule
\end{tabular}
\end{table}
\subsection{算法设计}
由于大部分基本操作都较为简单,因此在这里就不一一列举。
只给出插入 (\texttt{insert})和删除 (\texttt{delete})元素的算法设计。
\newline
\begin{algorithm}[H]
    \SetAlgoLined
    \KwIn{elem, index}
    \KwOut{none}
    \If{index out of range}{ throw error}
    \If{length == size}{ resize the list }
    \For{i=length; i\ge{index}; i++}{
        move the i th element backword
    }
    length++;
    \\
    elements[i]=elem
\caption{Insert}\label{alg:insert}
\end{algorithm}
插入算法如上述,其时间复杂度为$O(n)$,空间复杂度为$O(1)$
\begin{algorithm}
    \SetAlgoLined
    \KwIn{index}
    \KwOut{out}
    \If{index out of range}{ throw error }
    out=elements[index]
    \\
    \For{i=index; i < length; i++}{
        move the i th element forward
    }
    length--
\caption{Delete}\label{alg:delete}
\end{algorithm}
\newline
删除算法如上述,其时间复杂度为$O(n)$,空间复杂度为$O(1)$
\section{系统实现}
\subsection{开发环境}
本次实验中使用的环境如下:
\begin{enumerate}
    \item 操作系统版本 Darwin X86\_64 Kernel Version 18.7.0
    \item 编译器及其版本 clang++ version 10.0.1 (Apple LLVM version 10.0.1)
    \item 自动编译工具 CMake version 3.15.4
    \item 编程环境 NeoVim
\end{enumerate}
同时,本次实验的一部分代码是在另一环境下编写并测试通过:
\begin{enumerate}
    \item 操作系统及其版本:Arch Linux 5.3.7.arch1-2 (x86\_64)
    \item 编译器及其版本 clang++ version 9.0.0
    \item 自动编译工具 CMake version 3.15.4
    \item 编程环境 Visual Studio Code
\end{enumerate}
\subsection{代码结构及源代码}
本次实验采取了模块化的编码方式,
\footnote{具体代码结构见\autoref{appendix:structure}}
将线性表的不同部分的功能放置在不同源文件中,
\footnote{源代码见\autoref{appendix:lab1},测试代码见\autoref{appendix:test1}}
分别编译后链接,提高了代码的可维护性和编译速度,同时也更加容易编写单元测试,保证代码的正确性。
\subsection{代码亮点}
\begin{enumerate}
        \item 所有的代码均采用\emph{Google
            C/C++}标准代码规范以及\emph{ISOC++}委员会所推荐的\emph{CPP Core Guidelines}代码规范,同时使用\texttt{clang-format}和\texttt{clang-tidy}对代码进行格式化和规范化,符合现代化C++的规范。
        \item 所有代码均尽可能采用更新的C++标准,以提高可维护性以及可读性。
        \item 采用智能指针\footnote{需要C++11及以上的编译器支持}(\texttt{unique\_ptr})进行资源管理,避免了手动管理内存所可能造成的内存泄露等问题。
        \item 所有函数及变量均采用\texttt{auto}关键字进行声明,避免了错误的类型声明以及隐式的类型转换。
        \item 采用C++的模板类(\texttt{template})来实现线性表,体现出了线性表中能存放任何类型的元素、但同一线性表中只能存放单一类型元素的特点。
        \item 采用\texttt{CMake}自动生成\texttt{Makefile},从而简化了构建过程,同时能够实现跨平台编译。
        \item 使用现代化的C++测试框架\texttt{Catch2},简化了测试流程。
        \item 采用TDD\footnote{Test Driven Development}的开发方式,在保证效率的情况下尽可能地提高测试覆盖度,使系统更健壮,更容易维护。
\end{enumerate}

\subsection{系统测试}
本系统使用了\textit{Catch2}\footnote{url: https://github.com/catchorg/Catch2}作为测试框架,
对所有源代码编写了\textbf{完善}的单元测试,可以做到所有边界情况和越界情况以及正常情况全部覆盖,
同时程序中的每一个分支也都进行测试,使得测试覆盖度基本达到100\%。
% Please add the following required packages to your document preamble:
\subsubsection{构造函数测试}
\textbf{输入:}\texttt{size}变量,代表创建的线性表大小。
\par
\textbf{输出:}\texttt{list.size()},代表线性表实际大小。
\par
\textbf{预计结果:}两者相等。
\begin{table}[h]
\centering
\caption{构造函数测试}
\begin{tabular}{@{}ccccc@{}}
\toprule
\multicolumn{1}{c}{测试类型}    & \multicolumn{1}{c}{测试输入} & \multicolumn{1}{c}{理论输出} & \multicolumn{1}{c}{实际输出} & \multicolumn{1}{c}{测试结果} \\ \midrule
\multicolumn{1}{c|}{正确性测试}  & 10&10&10&正确\\
\multicolumn{1}{c|}{正确性测试}  & 空(默认构造函数)&1&1&正确\\
\multicolumn{1}{c|}{错误处理测试} & 0& std::underflow\_error& std::underflow\_error& 正确\\ \bottomrule
\end{tabular}
\label{tab:inittest1}
\end{table}

\subsubsection{判定空表函数测试}
\textbf{输入:}\texttt{list}变量,代表一个线性表。
\par
\textbf{输出:}\texttt{list.empty()},代表线性表是否为空。
\par
\textbf{预计结果:}对于空表,输出\texttt{true},否则输出\texttt{false}。
\begin{table}[h]
\centering
    \caption{\texttt{判定空表函数测试}}
\begin{tabular}{@{}ccccc@{}}
\toprule
\multicolumn{1}{c}{测试类型}    & \multicolumn{1}{c}{测试输入} & \multicolumn{1}{c}{理论输出} & \multicolumn{1}{c}{实际输出} & \multicolumn{1}{c}{测试结果} \\ \midrule
\multicolumn{1}{c|}{正确性测试}  & 空list&true&true&正确\\
\multicolumn{1}{c|}{正确性测试}  & 非空list&false&false&正确\\ \bottomrule
\end{tabular}
\label{tab:emptytest1}
\end{table}


\subsubsection{销毁表测试}
\textbf{输入:}无
\par
\textbf{输出:}无
\par
\textbf{预计结果:}表被销毁
\begin{table}[h]
\caption{销毁表测试}
\centering
\begin{tabular}{@{}ccccc@{}}
\toprule
\multicolumn{1}{c}{测试类型}    & \multicolumn{1}{c}{测试输入} & \multicolumn{1}{c}{理论输出} & \multicolumn{1}{c}{实际输出} & \multicolumn{1}{c}{测试结果} \\ \midrule
\multicolumn{1}{c|}{正确性测试}  & 无&表被销毁&表被销毁&正确\\ \bottomrule
\end{tabular}
\label{tab:destorytest1}
\end{table}

\subsubsection{清空表测试}
\textbf{输入:}无
\par
\textbf{输出:}无
\par
\textbf{预计结果:}表被清空
\begin{table}[h]
    \centering
    \caption{清空表测试}
    \begin{tabular}{@{}ccccc@{}}
        \toprule
        \multicolumn{1}{c}{测试类型}    & \multicolumn{1}{c}{测试输入} & \multicolumn{1}{c}{理论输出} & \multicolumn{1}{c}{实际输出} &
        \multicolumn{1}{c}{测试结果} \\ \midrule
        \multicolumn{1}{c|}{正确性测试}  &无 &表被清空&表被清空&正确\\ \bottomrule
    \end{tabular}
    \label{tab:cleartest1}
\end{table}


\subsubsection{求表长测试}
\textbf{输入:}一个线性表
\par
\textbf{输出:}\texttt{length},代表线性表实际大小。
\par
\textbf{预计结果:}输出与实际表长相等。
\begin{table}[h]
    \centering
    \caption{求表长测试}
    \begin{tabular}{@{}ccccc@{}}
        \toprule
        \multicolumn{1}{c}{测试类型}    & \multicolumn{1}{c}{测试输入} & \multicolumn{1}{c}{理论输出} & \multicolumn{1}{c}{实际输出} &
        \multicolumn{1}{c}{测试结果} \\ \midrule
        \multicolumn{1}{c|}{正确性测试}  & 长为10的线性表&10&10&正确\\
        \multicolumn{1}{c|}{正确性测试}  & 空线性表&0&0&正确\\ \bottomrule
    \end{tabular}
    \label{tab:lengthtest1}
\end{table}

\subsubsection{获得元素测试}
\textbf{输入:}\texttt{index}变量,代表想要获取元素的索引。
\par
\textbf{输出:}\texttt{element},代表获得到的元素。
\par
\textbf{预计结果:}想要获取到的元素与实际获取的元素相等。
\begin{table}[h]
    \centering
    \caption{获得元素测试}
    \begin{tabular}{@{}ccccc@{}}
        \toprule
        \multicolumn{1}{c}{测试类型}    & \multicolumn{1}{c}{测试输入} & \multicolumn{1}{c}{理论输出} & \multicolumn{1}{c}{实际输出} &
        \multicolumn{1}{c}{测试结果} \\ \midrule
        \multicolumn{1}{c|}{正确性测试}  & 0&``aaa''&``aaa''&正确\\
        \multicolumn{1}{c|}{正确性测试}  & 1&``bbb''&``bbb''&正确\\
        \multicolumn{1}{c|}{正确性测试}  & 3&``ddd''&``ddd''&正确\\
        \multicolumn{1}{c|}{错误处理测试} & 10000& std::overflow\_error& std::overflow\_error& 正确\\ \bottomrule
    \end{tabular}
    \label{tab:gettest1}
\end{table}

\subsubsection{查找元素测试}
\textbf{输入:}\texttt{element}变量,代表想要查找的元素。
\par
\textbf{输出:}\texttt{index},代表找到的元素的位置,没找到时返回-1。
\par
\textbf{预计结果:}与预期相等。
\begin{table}[h]
    \centering
    \caption{查找元素测试}
    \begin{tabular}{@{}ccccc@{}}
        \toprule
        \multicolumn{1}{c}{测试类型}    & \multicolumn{1}{c}{测试输入} & \multicolumn{1}{c}{理论输出} & \multicolumn{1}{c}{实际输出} &
        \multicolumn{1}{c}{测试结果} \\ \midrule
        \multicolumn{1}{c|}{正确性测试}  & ``aaa''&0&0&正确\\
        \multicolumn{1}{c|}{正确性测试}  & ``bbb''&1&1&正确\\
        \multicolumn{1}{c|}{正确性测试}  & ``magic''&-1&-1&正确\\ \bottomrule
    \end{tabular}
    \label{tab:locatetest1}
\end{table}


\subsubsection{获得前驱测试}
\textbf{输入:}\texttt{element}变量,代表想要搜索的位置。
\par
\textbf{输出:}\texttt{out}变量,代表搜索到的前驱元素。
\par
\textbf{预计结果:}搜到的元素与预期相等。
\begin{table}[h]
    \centering
    \caption{获得前驱测试}
    \begin{tabular}{@{}ccccc@{}}
        \toprule
        \multicolumn{1}{c}{测试类型}    & \multicolumn{1}{c}{测试输入} & \multicolumn{1}{c}{理论输出} & \multicolumn{1}{c}{实际输出} &
        \multicolumn{1}{c}{测试结果} \\ \midrule
        \multicolumn{1}{c|}{正确性测试}  & ``ddd''&``ccc''&``ccc''&正确\\
        \multicolumn{1}{c|}{错误处理测试} & ``aaa''& std::underflow\_error& std::underflow\_error& 正确\\ \bottomrule
    \end{tabular}
    \label{tab:priortest1}
\end{table}

\subsubsection{获得后继测试}
\textbf{输入:}\texttt{element}变量,代表想要搜索的位置。
\par
\textbf{输出:}\texttt{out}变量,代表搜索到的前驱元素。
\par
\textbf{预计结果:}搜到的元素与预期相等。
\begin{table}[h]
    \centering
    \caption{获得后继测试}
    \begin{tabular}{@{}ccccc@{}}
        \toprule
        \multicolumn{1}{c}{测试类型}    & \multicolumn{1}{c}{测试输入} & \multicolumn{1}{c}{理论输出} & \multicolumn{1}{c}{实际输出} &
        \multicolumn{1}{c}{测试结果} \\ \midrule
        \multicolumn{1}{c|}{正确性测试}  & ``ccc ''&``ddd ''&``ddd ''&正确\\
        \multicolumn{1}{c|}{错误处理测试} & 0& std::overflow\_error& std::overflow\_error& 正确\\ \bottomrule
    \end{tabular}
    \label{tab:nexttest1}
\end{table}

\subsubsection{插入元素测试}
\textbf{输入:}\texttt{(index, element)}变量,代表要插入的位置和要插入的元素。
\par
\textbf{输出:}\texttt{(length, element)},代表线性表插入后的长度和插入位置的元素。
\par
\textbf{预计结果:}两者对应相等。
\begin{table}[h]
    \centering
    \caption{插入元素测试}
    \begin{tabular}{@{}ccccc@{}}
        \toprule
        \multicolumn{1}{c}{测试类型}    & \multicolumn{1}{c}{测试输入} & \multicolumn{1}{c}{理论输出} & \multicolumn{1}{c}{实际输出} &
        \multicolumn{1}{c}{测试结果} \\ \midrule
        \multicolumn{1}{c|}{正确性测试}  & (1, ``sss '')(从头插入)&(length+1, ``sss'')&(length+1, ``sss'')&正确\\
        \multicolumn{1}{c|}{正确性测试}  & (length, `` sss '')(从尾部插入)&(length+1, ``sss'')&(length+1, ``sss'')&正确\\
        \multicolumn{1}{c|}{错误处理测试} & 0& std::underflow\_error& std::underflow\_error& 正确\\
        \multicolumn{1}{c|}{错误处理测试} & 100& std::overflow\_error& std::overflow\_error& 正确\\ \bottomrule
    \end{tabular}
    \label{tab:inserttest1}
\end{table}

\subsubsection{删除元素测试}
\textbf{输入:}\texttt{index}变量,代表要删除的位置。
\par
\textbf{输出:}\texttt{(length, element)},代表线性表删除后的长度和删除后该位置的元素。
\par
\textbf{预计结果:}两者对应相等。
\begin{table}[h]
    \centering
    \caption{删除元素测试}
    \begin{tabular}{@{}ccccc@{}}
        \toprule
        \multicolumn{1}{c}{测试类型}    & \multicolumn{1}{c}{测试输入} & \multicolumn{1}{c}{理论输出} & \multicolumn{1}{c}{实际输出} &
        \multicolumn{1}{c}{测试结果} \\ \midrule
        \multicolumn{1}{c|}{正确性测试}  & 1(从头删除)&(length-1, ``bbb'')&(length-1, ``bbb'')&正确\\
        \multicolumn{1}{c|}{正确性测试}  & (length, `` bbb'')(从尾部删除)&(length-1, ``bbb'')&(length-1, ``bbb'')&正确\\
        \multicolumn{1}{c|}{错误处理测试} & 0& std::underflow\_error& std::underflow\_error& 正确\\
        \multicolumn{1}{c|}{错误处理测试} & 100& std::overflow\_error& std::overflow\_error& 正确\\ \bottomrule
    \end{tabular}
    \label{tab:deletetest1}
\end{table}


\subsubsection{遍历表测试}
\textbf{输入:}\texttt{visit}变量,代表要遍历执行的函数。
\par
\textbf{输出:}\texttt{list}变量,代表新的list线性表。
\par
\textbf{预计结果:}两者对应相等。
\begin{table}[h]
    \caption{遍历表测试}
    \centering
    \begin{tabular}{@{}ccccc@{}}
        \toprule
        \multicolumn{1}{c}{测试类型}    & \multicolumn{1}{c}{测试输入} & \multicolumn{1}{c}{理论输出} & \multicolumn{1}{c}{实际输出} &
        \multicolumn{1}{c}{测试结果} \\ \midrule
        \multicolumn{1}{c|}{正确性测试}  & 将每个元素乘2&(2,4,6,8)&(2,4,6,8)&正确\\ \bottomrule
    \end{tabular}
    \label{tab:traversetest1}
\end{table}

\subsubsection{测试小结}
所有测试均通过,并且绝大部分的边界条件也都覆盖到了。可以认为程序中不存在逻辑错误,
了所有基本功能,并且能够执行文件操作,同时也可以在客户端进行多表操作。
\par
通过测试,证明了系统实现的完整性和正确性,确保了系统的良好运行。

\section{实验结果与分析}\label{sec:4}
本次实验加深了对线性表的概念、基本运算的理解,
掌握了线性表的基本运算的实现。
深刻理解了线性表的\emph{逻辑结构}和\emph{物理结构}之间的关系。
\par
作为第一次实验,本次实验的内容还是相对比较简单的,通过整个编程过程,我又熟悉了一遍顺序结构的线性表,对原先的知识进行了一次很好的复习。
\par 整个实验均在\texttt{UNIX}环境下编程(先是ArchLinux,后是macOS),
所有的代码均采用\emph{Google C/C++}标准代码规范,
通过\texttt{clang-format}和\texttt{clang-tidy}进行格式化和规范化。
\par
同时,在编码过程中,我尽可能地使用了更加现代化的C++代码,例如使用智能指针(\textit{unique\_ptr})来进行资源管理,从而避免了手动管理内存而可能带来的内存泄漏等问题、
例如使用\texttt{auto}关键字来声明函数与变量,从而减少了错误的类型声明或不正确的隐式类型转换。
使得代码可读性更高,更容易维护,更加健壮。这样的规范和编码习惯有助于以后在工作中更高效地完成工作任务。
\par
本系统完整的实现了课程要求的全部功能,并且实现了多线性表管理和文件存储功能,
系统健壮性良好,可以应对各种情况的输入,且能输出相应的错误提示。
系统测试覆盖度接近100\%,可以认为不会发生逻辑错误
\par
今后的学习过程中,应当多从数据结构的角度分析如何进行数据的处理、存储以方便问题的解决,并 要勤加练习达到熟能生巧的地步。


\include{lab2}
\chapter{基于线性存储结构的二叉树实现}\label{chapter:3}

\section{实验目的}\label{sec:test31}
通过实验达到
\begin{itemize}
    \item 加深对二叉树的概念、基本运算的理解
    \item 熟练掌握二叉树的逻辑结构与物理结构的关系
    \item 以二叉链二叉树作为物理结构,熟练掌握二叉树基本运算的实现。
\end{itemize}

\subsection{对二叉树对理解}
通过本次实验,我深刻理解了二叉树的\textbf{二叉}的意义,以及其作为一棵树的特点。即所有节点均\textbf{像一棵树一样}地被组织在一起,每个节点可以有父亲和两个孩子。
\newline
\subsection{对基本运算的理解与实现}
总的来说,二叉树的基本运算较简单,主要的难点在创建(\texttt{create})插入 (\texttt{insert})节点与删除 (\texttt{delete})节点。
因为需要读取定义的内容对二叉树的长度 (\texttt{length})进行改变,
同时要分配新的内存并创建新的节点。
如果操作不当,或者没有对用户的输入进行校验,插入或删除出错。
\subsection{单元测试}
通过编写完整的单元测试,将可能出现的错误都考虑清楚,尽量实现测试覆盖率达到100\%。
从而保证了程序在正确情况和极端情况下都能正常运行。

% ------------------------------------------------------------

\section{实验内容}\label{sec:test32}
    实验内容主要分为一下三个部分:
\begin{enumerate}
    \item 问题描述
    \item 系统设计
    \item 系统实现
\end{enumerate}
\section{问题描述}
\subsection{二叉树的定义}
\begin{definition}\label{def:binarytree3}
    二叉树 (\emph{Binary Tree})是每个结点最多有两个子树的树结构,通常子树被称作“左子树”(left subtree)和“右子树”(right subtree)
\end{definition}
其中:
\begin{itemize}
    \item 数据节点可以为任意类型,但同一二叉树中节点类型必须相同。
    \item 数据节点的个数$n$定义为二叉树的长度 (\emph{length}),二叉树里没有一个节点时称为空二叉树。
    \item 对于非空的二叉树,每一个数据节点都有其确定的位置
    \item 树的高度也称为树的深度(\emph{depth})
    \item 而对于每一个数据节点,除了根结点外,均有父亲,而除了叶子节点外,均有孩子
\end{itemize}
通过本次实验,我理解到了二叉树的精髓所在:
通过数据节点\emph{逻辑}上的父子/兄弟关系来表达数据之间关系
\newline
二叉树的存储特点是:只要确定了\emph{父亲},孩子就可以很简单的通过指针来获得:
因此,二叉树可以通过逻辑上的相连关系来实现二叉树,而不需要连续的物理空间。
然而,二叉树无法对节点进行随机存储,因此在搜索和访问时会有$O(n)$的时间复杂度。
\subsection{实验需要完成的基本操作}
\begin{enumerate}
\item 创建二叉树:\texttt{CreateTree(T)},\newline \textbf{\emph{初始条件}}是二叉树T不存在已存在 。\newline \textbf{\emph{操作结果}}是构造一个空的二叉树。
\item 销毁二叉树:\texttt{DestroyTree(T)},\newline \textbf{\emph{初始条件}}是二叉树T已存在 。\newline \textbf{\emph{操作结果}}是销毁二叉树T。
\item 清空二叉树:\texttt{ClearTree(T)},\newline \textbf{\emph{初始条件}}是二叉树T已存在 。\newline \textbf{\emph{操作结果}}是将T重置为空二叉树。
\item 判定空二叉树:\texttt{TreeEmpty(T)},\newline \textbf{\emph{初始条件}}是二叉树T已存在 。\newline \textbf{\emph{操作结果}}是若T为空二叉树则返回\textbf{TRUE},否则返回\textbf{FALSE}。
\item 求二叉树深度:\texttt{TreeDepth(T)},\newline \textbf{\emph{初始条件}}是二叉树已存在 。\newline \textbf{\emph{操作结果}}是返回T的深度
\item 查找节点:\texttt{LocateNode(T,e)},\newline \textbf{\emph{初始条件}}是二叉树已存在 。\newline
    \textbf{\emph{操作结果}}返回查找到的结点指针,如无关键字为e的结点,返回NULL。
\item 节点赋值:\texttt{AssginNode(T,e,value)},\newline \textbf{\emph{初始条件}}是二叉树T已存在 。\newline \textbf{\emph{操作结果}}是关键字为e的结点赋值为value。
\item 获得兄弟:\texttt{GetSibling(T,e)},\newline \textbf{\emph{初始条件}}是二叉树T已存在 。\newline \textbf{\emph{操作结果}}返回关键字为e结点的(左或右)兄弟结点指针。若关键字为e的结点无兄弟,则返回NULL。
\item 插入节点:\texttt{InsertNode(T,e,lr,c)},\newline \textbf{\emph{初始条件}}是二叉树T已存在且非空,1\le i \le TreeLength(T)+1 。\newline \textbf{\emph{操作结果}}是根据LR为0或者1,插入结点c到T中,作为关键字为e的结点的左或右孩子结点,结点e的原有左子树或右子树则为结点c的右子树
\item 删除节点:\texttt{DeleteNode(T,i,e)},\newline \textbf{\emph{初始条件}}是二叉树T已存在且非空,1\le i\le TreeLength(T) 。\newline \textbf{\emph{操作结果}}:是删除T中关键字为e的结点;同时,如果关键字为e的结点度为0,删除即可;如关键字为e的结点度为1,用关键字为e的结点孩子代替被删除的e位置;如关键字为e的结点度为2,用e的左孩子代替被删除的e位置,e的右子树作为e的左子树中最右结点的右子树。
\item 前序遍历:\texttt{PreOrderTraverse(T,visit())},\newline \textbf{\emph{初始条件}}是二叉树T已存在,\newline \textbf{\emph{操作结果}}根据前序对T的每个数据节点调用函数visit()。
\item 中序遍历:\texttt{PreOrderTraverse(T,visit())},\newline \textbf{\emph{初始条件}}是二叉树T已存在,\newline \textbf{\emph{操作结果}}根据中序对T的每个数据节点调用函数visit()。
\item 后序遍历:\texttt{PreOrderTraverse(T,visit())},\newline \textbf{\emph{初始条件}}是二叉树T已存在,\newline \textbf{\emph{操作结果}}根据后序对T的每个数据节点调用函数visit()。
\item 层级遍历:\texttt{PreOrderTraverse(T,visit())},\newline \textbf{\emph{初始条件}}是二叉树T已存在,\newline \textbf{\emph{操作结果}}根据层级对T的每个数据节点调用函数visit()。
\end{enumerate}
\section{系统设计}
\subsection{总体设计}
本系统采用\emph{二叉链表}作为二叉树的物理结构,实现二叉树的基本运算。
\par
系统开始运行的时候默认不使用文件中的数据,但是用户随时可以将文件中的数据导入到内存中,同时提供数据保存的功能。
\subsection{有关常量和类型定义}
采取\texttt{C++}中的模版类来使二叉树支持所有类型的数据,
为防止手动管理内存而造成内存泄露的问题,
采用\texttt{unique\_ptr}\footnote{需要C++11及以上的编译器支持}对底层的指针进行管理。
\par
此外,二叉树类中还有个成员变量,\texttt{\_length},
代表当前二叉树的已有节点数量
\par
另外,作为封装,将\texttt{\_length}声明为私有成员,防止被非友元函数篡改。
对于程序中可能出现的错误,进行统一规定:
\begin{enumerate}
    \item 对于用户输入不正确导致的在没有结点处删除,统一抛出\texttt{range\_error}。
    \item 对于其他可能发生的错误,统一抛出\texttt{runtime\_error}。
\end{enumerate}
\subsubsection{函数设计}
头文件中的函数原型声明见\autoref{appendix:h3}
\par
函数时间复杂度分析见\autoref{tab:timeandspace3}
\begin{table}[Htb]
\centering
\caption{函数时间与空间复杂度分析}
\label{tab:timeandspace3}
\begin{tabular}{@{}ccccc@{}}
\toprule
编号                          & 名称  & 函数签名 & 时间复杂度 & 空间复杂度 \\ \toprule
    \multicolumn{1}{c|}{1}  & 创建二叉树 & \texttt{ createTree()} & $O(n)$ &  $O(n)$ \\
    \multicolumn{1}{c|}{2}  & 销毁二叉树& \texttt{ destroyTree()} & $O(n)$ &  $O(n)$ \\
    \multicolumn{1}{c|}{3}  & 清空二叉树& \texttt{ clearTree()} & $O(1)$ &  $O(1)$   \\
    \multicolumn{1}{c|}{4}  & 判定空二叉树& \texttt{ empty()} & $O(1)$ &  $O(1)$     \\
    \multicolumn{1}{c|}{5}  & 求二叉树深度 & \texttt{ depth()} & $O(n)$ &  $O(n)$     \\
    \multicolumn{1}{c|}{6}  & 节点赋值 & \texttt{ assign()} & $O(n)$ &  $O(n)$      \\
    \multicolumn{1}{c|}{7}  & 查找节点 & \texttt{ locate()} & $O(n)$ &  $O(n)$   \\
    \multicolumn{1}{c|}{8}  & 获得兄弟 & \texttt{ sibling()} & $O(n)$ &  $O(n)$    \\
    \multicolumn{1}{c|}{9}  & 获得父亲 & \texttt{ father()} & $O(n)$ &  $O(n)$     \\
    \multicolumn{1}{c|}{10}  & 插入节点 & \texttt{ insert()} & $O(1)$ &  $O(n)$  \\
    \multicolumn{1}{c|}{11}  & 删除节点 & \texttt{ delete()} & $O(n)$ &  $O(n)$  \\
    \multicolumn{1}{c|}{12}  & 遍历二叉树 & \texttt{ traverse()}   & $O(n)$ & $O(n)$ \\ \bottomrule
\end{tabular}
\end{table}
\subsubsection{算法设计}
\paragraph{创建算法}如\autoref{alg:create3}所述,每次找到根结点后,递归地创建子树
\par
\begin{algorithm}[H]
    \SetAlgoLined
    \KwIn{postDef, inDef}
    \KwOut{Tree}
    \If{postDef.length != inDef.length}{throw error}
    set index for every key of postDef in inDef\;
    let root = postDef[length-1]\;
    let rootIndex = index of last in inDef\;
    let nextRoot = postDef[length-2]\;
    \If{rootIndex is not last}{create right tree with nextRoot\; create left tree if has}
    \Else{create left tree with nextRoot}
    \caption{Create}\label{alg:create3}
\end{algorithm}
\paragraph{插入算法}如\autoref{alg:insert3}所述,其时间复杂度为$O(n)$,空间复杂度为$O(n)$
\par
\begin{algorithm}[H]
    \SetAlgoLined
    \KwIn{elem, index}
    \KwOut{none}
    \If{index out of range}{ throw error}
    let ptr = first pointer\;
    \For{i=index; i\ge{1}; i--}{
        move the ptr backword
    }
    new\_node = Node(elem,ptr.next)\;
    ptr.next = new\_node\;
    length++
\caption{Insert}\label{alg:insert3}
\end{algorithm}
\paragraph{删除算法}如\autoref{alg:delete3}所述,其时间复杂度为$O(n)$,空间复杂度为$O(n)$
\par
\begin{algorithm}[H]
    \SetAlgoLined
    \KwIn{index}
    \KwOut{out}
    \If{index out of range}{ throw error }
    let out = empty element\;
    let ptr = first pointer\;
    \For{i=index; i > 1; i--}{
        move the ptr backword
    }
    ptr.next = ptr.next.next\;
    length--
\caption{Delete}\label{alg:delete3}
\end{algorithm}
\paragraph{求深度算法}如\autoref{alg:depth3}所述,其时间复杂度为$O(n)$,空间复杂度为$O(n)$
\par
\begin{algorithm}[H]
    \SetAlgoLined
    \KwIn{T}
    \KwOut{depth}
    \SetKwFunction{Max}{Max}
    \SetKwFunction{Depth}{Depth}
    let leftDepth = 0\;
    let rightDepth = 0\;
    \If(\tcp*[f]{has no child}){T.leftChild ==nullptr \textbf{And} T.rightChild == nullptr}{%
        \Return{$0$}\;
    }
    \eIf{T.leftChild!=nullptr}{
        \tcp{has left child}
        leftDepth = {\Depth(T.leftChild)}\;
    }{\tcp{has right child}
        rightDepth = {\Depth(T.rightChild)}\;
    }
    \If{T.rightChild !=nullptr}{rightChild = {\Depth(T.rightChild)}\;}

    \Return{\Max(rightDepth,leftDepth)}
\caption{Depth}\label{alg:depth3}
\end{algorithm}

\section{系统实现}
\subsection{开发环境}
本次实验中使用的环境如下:
\begin{enumerate}
    \item 操作系统版本 Darwin X86\_64 Kernel Version 18.7.0
    \item 编译器及其版本 clang++ version 10.0.1 (Apple LLVM version 10.0.1)
    \item 自动编译工具 CMake version 3.15.4
    \item 编程环境 NeoVim
\end{enumerate}
\subsection{代码结构及源代码}
本次实验采取了模块化的编码方式,
\footnote{具体代码结构见\autoref{appendix:structure}}
将二叉树的不同部分的功能放置在不同源文件中,
\footnote{源代码见\autoref{appendix:lab2},测试代码见\autoref{appendix:test3}}
分别编译后链接,提高了代码的可维护性和编译速度,同时也更加容易编写单元测试,保证代码的正确性。
\subsection{代码亮点}
\begin{enumerate}
        \item 所有的代码均采用\emph{Google
            C/C++}标准代码规范以及\emph{ISOC++}委员会所推荐的\emph{CPP Core Guidelines}代码规范,同时使用\texttt{clang-format}和\texttt{clang-tidy}对代码进行格式化和规范化,符合现代化C++的规范。
        \item 所有代码均尽可能采用更新的C++标准,以提高可维护性以及可读性。
        \item 采用智能指针\footnote{需要C++11及以上的编译器支持}(\texttt{unique\_ptr})进行资源管理,避免了手动管理内存所可能造成的内存泄露等问题。
        \item 所有函数及变量均采用\texttt{auto}关键字进行声明,避免了错误的类型声明以及隐式的类型转换。
        \item 采用C++的模板类(\texttt{template})来实现二叉树,体现出了二叉树中能存放任何类型的节点、但同一二叉树中只能存放单一类型节点的特点。
        \item 采用\texttt{CMake}自动生成\texttt{Makefile},从而简化了构建过程,同时能够实现跨平台编译。
        \item 使用现代化的C++测试框架\texttt{Catch2},简化了测试流程。
        \item 采用TDD\footnote{Test Driven Development}的开发方式,在保证效率的情况下尽可能地提高测试覆盖度,使系统更健壮,更容易维护。
        \item 使用\texttt{Sanitize}与\texttt{Valgrind}对代码进行检测,发现可能存在的内存泄漏或未定义行为。
        \item 使用\texttt{gcov, lcov, llvm-cov gcov}等工具对单元测试的覆盖度进行检测,并生成相应的测试报告。
\end{enumerate}

\subsection{系统测试}
本系统使用了\textit{Catch2}\footnote{url: https://github.com/catchorg/Catch2}作为测试框架,
对所有源代码编写了\textbf{完善}的单元测试,可以做到所有边界情况和越界情况以及正常情况全部覆盖,
同时程序中的每一个分支也都进行测试,使得测试覆盖度基本达到100\%。
% Please add the following required packages to your document preamble:
\subsubsection{构造函数测试}
\textbf{输入:}\texttt{postDef,inDef}变量,创建的二叉树定义。
\par
\textbf{输出:}\texttt{std::vector<T> res},二叉树前序遍历的结果。
\par
\textbf{预计结果:}前序遍历正确。
\begin{table}[Htb]
\centering
\caption{构造函数测试}
\begin{tabular}{@{}ccccc@{}}
\toprule
\multicolumn{1}{c}{测试类型}    & \multicolumn{1}{c}{测试输入} & \multicolumn{1}{c}{理论输出} & \multicolumn{1}{c}{实际输出} & \multicolumn{1}{c}{测试结果} \\ \midrule
\multicolumn{1}{c|}{正确性测试}  & 随机一棵树的Def&前序遍历&前序遍历&正确\\
\multicolumn{1}{c|}{正确性测试}  & 空(默认构造函数)&1&1&正确\\
\multicolumn{1}{c|}{边界条件测试}  & 单枝树&前序遍历&前序遍历&正确\\
\multicolumn{1}{c|}{错误处理测试} & 错误定义& std::underflow\_error& std::underflow\_error& 正确\\ \bottomrule
\end{tabular}
\label{tab:inittest3}
\end{table}

\subsubsection{判定空二叉树函数测试}
\textbf{输入:}\texttt{tree}变量,一个二叉树。
\par
\textbf{输出:}\texttt{tree.empty()},二叉树是否为空。
\par
\textbf{预计结果:}对于空二叉树,输出\texttt{true},否则输出\texttt{false}。
\begin{table}[Htb]
\centering
    \caption{\texttt{判定空二叉树函数测试}}
\begin{tabular}{@{}ccccc@{}}
\toprule
\multicolumn{1}{c}{测试类型}    & \multicolumn{1}{c}{测试输入} & \multicolumn{1}{c}{理论输出} & \multicolumn{1}{c}{实际输出} & \multicolumn{1}{c}{测试结果} \\ \midrule
\multicolumn{1}{c|}{正确性测试}  & 空tree&true&true&正确\\
\multicolumn{1}{c|}{正确性测试}  & 非空tree&false&false&正确\\ \bottomrule
\end{tabular}
\label{tab:emptytest3}
\end{table}


\subsubsection{销毁二叉树测试}
\textbf{输入:}无
\par
\textbf{输出:}无
\par
\textbf{预计结果:}二叉树被销毁
\begin{table}[Htb]
\caption{销毁二叉树测试}
\centering
\begin{tabular}{@{}ccccc@{}}
\toprule
\multicolumn{1}{c}{测试类型}    & \multicolumn{1}{c}{测试输入} & \multicolumn{1}{c}{理论输出} & \multicolumn{1}{c}{实际输出} & \multicolumn{1}{c}{测试结果} \\ \midrule
\multicolumn{1}{c|}{正确性测试}  & 无&二叉树被销毁&二叉树被销毁&正确\\ \bottomrule
\end{tabular}
\label{tab:destorytest3}
\end{table}


\subsubsection{清空二叉树测试}
\textbf{输入:}无
\par
\textbf{输出:}无
\par
\textbf{预计结果:}二叉树被清空
\begin{table}[Htb]
    \centering
    \caption{清空二叉树测试}
    \begin{tabular}{@{}ccccc@{}}
        \toprule
        \multicolumn{1}{c}{测试类型}    & \multicolumn{1}{c}{测试输入} & \multicolumn{1}{c}{理论输出} & \multicolumn{1}{c}{实际输出} &
        \multicolumn{1}{c}{测试结果} \\ \midrule
        \multicolumn{1}{c|}{正确性测试}  &无 &二叉树被清空&二叉树被清空&正确\\ \bottomrule
    \end{tabular}
    \label{tab:cleartest3}
\end{table}


\subsubsection{求二叉树深度测试}
\textbf{输入:}一个二叉树
\par
\textbf{输出:}\texttt{depth},二叉树实际深度。
\par
\textbf{预计结果:}输出与实际二叉树深度相等。
\begin{table}[Htb]
    \centering
    \caption{求二叉树长测试}
    \begin{tabular}{@{}ccccc@{}}
        \toprule
        \multicolumn{1}{c}{测试类型}    & \multicolumn{1}{c}{测试输入} & \multicolumn{1}{c}{理论输出} & \multicolumn{1}{c}{实际输出} &
        \multicolumn{1}{c}{测试结果} \\ \midrule
        \multicolumn{1}{c|}{正确性测试}  & 深度为5的二叉树&5&5&正确\\
        \multicolumn{1}{c|}{正确性测试}  & 空二叉树&0&0&正确\\ \bottomrule
    \end{tabular}
    \label{tab:lengthtest3}
\end{table}


\subsubsection{查找节点测试}
\textbf{输入:}\texttt{key}变量,要获取节点的key。
\par
\textbf{输出:}\texttt{element},获得到的节点。
\par
\textbf{预计结果:}想要获取到的节点与实际获取的节点相等。
\begin{table}[Htb]
    \centering
    \caption{查找节点测试}
    \begin{tabular}{@{}ccccc@{}}
        \toprule
        \multicolumn{1}{c}{测试类型}    & \multicolumn{1}{c}{测试输入} & \multicolumn{1}{c}{理论输出} & \multicolumn{1}{c}{实际输出} &
        \multicolumn{1}{c}{测试结果} \\ \midrule
        \multicolumn{1}{c|}{正确性测试}  & 0&``aaa''&``aaa''&正确\\
        \multicolumn{1}{c|}{正确性测试}  & 1&``bbb''&``bbb''&正确\\
        \multicolumn{1}{c|}{正确性测试}  & 3&``ddd''&``ddd''&正确\\
        \multicolumn{1}{c|}{错误处理测试} & 10000& std::overflow\_error& std::overflow\_error& 正确\\ \bottomrule
    \end{tabular}
    \label{tab:gettest3}
\end{table}

\subsubsection{节点赋值测试}
\textbf{输入:}\texttt{key, value}变量,想要赋值的节点的key和新的value。
\par
\textbf{输出:}\texttt{elem},新的节点的位置,没找到时返回-1。
\par
\textbf{预计结果:}与预期相等。
\begin{table}[Htb]
    \centering
    \caption{节点赋值测试}
    \begin{tabular}{@{}ccccc@{}}
        \toprule
        \multicolumn{1}{c}{测试类型}    & \multicolumn{1}{c}{测试输入} & \multicolumn{1}{c}{理论输出} & \multicolumn{1}{c}{实际输出} &
        \multicolumn{1}{c}{测试结果} \\ \midrule
        \multicolumn{1}{c|}{正确性测试}  & 1,``aaa''&0&0&正确\\
        \multicolumn{1}{c|}{正确性测试}  & 3,``bbb''&1&1&正确\\
        \multicolumn{1}{c|}{正确性测试}  & 1000,``magic''&-1&-1&正确\\ \bottomrule
    \end{tabular}
    \label{tab:locatetest3}
\end{table}


\subsubsection{获得兄弟测试}
\textbf{输入:}\texttt{e}变量,要搜索的位置。
\par
\textbf{输出:}\texttt{out}变量,搜索到的兄弟节点。
\par
\textbf{预计结果:}搜到的节点与预期相等。
\begin{table}[Htb]
    \centering
    \caption{获得兄弟测试}
    \begin{tabular}{@{}ccccc@{}}
        \toprule
        \multicolumn{1}{c}{测试类型}    & \multicolumn{1}{c}{测试输入} & \multicolumn{1}{c}{理论输出} & \multicolumn{1}{c}{实际输出} &
        \multicolumn{1}{c}{测试结果} \\ \midrule
        \multicolumn{1}{c|}{正确性测试}  & ``ddd''&``ccc''&``ccc''&正确\\
        \multicolumn{1}{c|}{错误处理测试} & ``aaa''& std::underflow\_error& std::underflow\_error& 正确\\ \bottomrule
    \end{tabular}
    \label{tab:priortest3}
\end{table}

\subsubsection{获得父亲测试}
\textbf{输入:}\texttt{e}变量,要搜索的位置。
\par
\textbf{输出:}\texttt{out}变量,搜索到的父亲。
\par
\textbf{预计结果:}搜到的节点与预期相等。
\begin{table}[Htb]
    \centering
    \caption{获得父亲测试}
    \begin{tabular}{@{}ccccc@{}}
        \toprule
        \multicolumn{1}{c}{测试类型}    & \multicolumn{1}{c}{测试输入} & \multicolumn{1}{c}{理论输出} & \multicolumn{1}{c}{实际输出} &
        \multicolumn{1}{c}{测试结果} \\ \midrule
        \multicolumn{1}{c|}{正确性测试}  & ``ccc ''&``ddd ''&``ddd ''&正确\\
        \multicolumn{1}{c|}{错误处理测试} & 0& std::overflow\_error& std::overflow\_error& 正确\\ \bottomrule
    \end{tabular}
    \label{tab:nexttest3}
\end{table}

\subsubsection{插入节点测试}
\textbf{输入:}\texttt{(LR, element, c)}变量,要插入的位置和要插入的节点。
\par
\textbf{输出:}\texttt{(element)},插入的节点。
\par
\textbf{预计结果:}两者对应相等。
\begin{table}[Htb]
    \centering
    \caption{插入节点测试}
    \begin{tabular}{@{}ccccc@{}}
        \toprule
        \multicolumn{1}{c}{测试类型}    & \multicolumn{1}{c}{测试输入} & \multicolumn{1}{c}{理论输出} & \multicolumn{1}{c}{实际输出} &
        \multicolumn{1}{c}{测试结果} \\ \midrule
        \multicolumn{1}{c|}{正确性测试}  & (1, ``sss '')(从头插入)&``sss''& ``sss''&正确\\
        \multicolumn{1}{c|}{正确性测试}  & (length, `` sss '')(从尾部插入)&``sss''& ``sss''&正确\\
        \multicolumn{1}{c|}{错误处理测试} & 0& std::underflow\_error& std::underflow\_error& 正确\\
        \multicolumn{1}{c|}{错误处理测试} & 100& std::overflow\_error& std::overflow\_error& 正确\\ \bottomrule
    \end{tabular}
    \label{tab:inserttest3}
\end{table}


\subsubsection{删除节点测试}
\textbf{输入:}\texttt{elem}变量,要删除的位置。
\par
\textbf{输出:}\texttt{(length, element)},二叉树删除后的长度和删除后该位置的节点。
\par
\textbf{预计结果:}两者对应相等。
\begin{table}[Htb]
    \centering
    \caption{删除节点测试}
    \begin{tabular}{@{}ccccc@{}}
        \toprule
        \multicolumn{1}{c}{测试类型}    & \multicolumn{1}{c}{测试输入} & \multicolumn{1}{c}{理论输出} & \multicolumn{1}{c}{实际输出} &
        \multicolumn{1}{c}{测试结果} \\ \midrule
        \multicolumn{1}{c|}{正确性测试}  & 1(从头删除)&(length-1, ``bbb'')&(length-1, ``bbb'')&正确\\
        \multicolumn{1}{c|}{正确性测试}  & (length, `` bbb'')(从尾部删除)&(length-1, ``bbb'')&(length-1, ``bbb'')&正确\\
        \multicolumn{1}{c|}{错误处理测试} & 0& std::underflow\_error& std::underflow\_error& 正确\\
        \multicolumn{1}{c|}{错误处理测试} & 100& std::overflow\_error& std::overflow\_error& 正确\\ \bottomrule
    \end{tabular}
    \label{tab:deletetest3}
\end{table}


\subsubsection{遍历二叉树测试}
\textbf{输入:}\texttt{visit}变量,要遍历执行的函数。
\par
\textbf{输出:}\texttt{tree}变量,新的tree二叉树。
\par
\textbf{预计结果:}两者对应相等。
\begin{table}[Htb]
    \caption{遍历二叉树测试}
    \centering
    \begin{tabular}{@{}ccccc@{}}
        \toprule
        \multicolumn{1}{c}{测试类型}    & \multicolumn{1}{c}{测试输入} & \multicolumn{1}{c}{理论输出} & \multicolumn{1}{c}{实际输出} &
        \multicolumn{1}{c}{测试结果} \\ \midrule
        \multicolumn{1}{c|}{正确性测试}  & 将每个节点乘2&(2,4,6,8)&(2,4,6,8)&正确\\ \bottomrule
    \end{tabular}
    \label{tab:traversetest3}
\end{table}

\subsubsection{测试小结}
所有测试均通过,并且绝大部分的边界条件也都覆盖到了。可以认为程序中不存在逻辑错误,
了所有基本功能,并且能够执行文件操作,同时也可以在客户端进行多二叉树操作。
\par
通过测试,证明了系统实现的完整性和正确性,确保了系统的良好运行。
\autoref{fig:cov3}中为测试覆盖度\footnote{更具体的可以见\texttt{out}文件夹}

\begin{figure}
\centering
\caption{测试覆盖度}\label{fig:cov3}
\includegraphics[scale=.5]{cov.png}
\end{figure}

\section{实验结果与分析}\label{sec:test34}
本次实验加深了对二叉树的概念、基本运算的理解,
掌握了二叉树的基本运算的实现。
深刻理解了二叉树的\emph{逻辑结构}和\emph{物理结构}之间的关系。
并通过与实验一的对比,了解到了二叉树不同的物理存储结构之间的相同点与不同点。
\par
作为第二次实验,本次实验的内容还是相对比较简单的,通过整个编程过程,我又熟悉了一遍链二叉树的实现,对原先的知识进行了一次很好的复习。
\par 整个实验均在\texttt{UNIX}环境下编程
所有的代码均采用\emph{Google C/C++}标准代码规范,
通过\texttt{clang-format}和\texttt{clang-tidy}进行格式化和规范化。
\par
同时,在编码过程中,我尽可能地使用了更加现代化的C++代码,例如使用智能指针(\textit{unique\_ptr})来进行资源管理,从而避免了手动管理内存而可能带来的内存泄漏等问题、
例如使用\texttt{auto}关键字来声明函数与变量,从而减少了错误的类型声明或不正确的隐式类型转换。
使得代码可读性更高,更容易维护,更加健壮。这样的规范和编码习惯有助于以后在工作中更高效地完成工作任务。
\par
本系统完整的实现了课程要求的全部功能,并且实现了多二叉树管理和文件存储功能,
系统健壮性良好,可以应对各种情况的输入,且能输出相应的错误提示。
系统测试覆盖度接近100\%,可以认为不会发生逻辑错误
\par


\chapter{基于邻接链表的图的实现}\label{chapter:4}

\section{实验目的}\label{sec:test41}
通过实验达到
\begin{itemize}
    \item 加深对图的概念、基本运算的理解
    \item 熟练掌握图的逻辑结构与物理结构的关系
    \item 以邻接链表作为物理结构,熟练掌握图基本运算的实现。
\end{itemize}

\subsection{对图对理解}
通过本次实验,我对\textbf{图}又有了更多的了解,由于图的结构比较复杂,任意两个节点之间都可能存在联系,因此无法以数据元素存储区中的物理位置来表示元素之间的关系,即图没有顺序映像的存储结构,但可以借助数组数据类型表示元素之间的关系。~\cite{严蔚敏2002数据结构}
\subsection{对邻接链表的理解}
\textbf{邻接链表}是图的一种链式存储结构。在邻接链表中,对图中每个顶点建立一个单链表,第$i$个单链表中的节点表示依附于顶点$v_{i}$的边(对有向图是以顶点$v_{i}$为尾的弧)。每个结点由三个域组成,其中邻接点域指示域顶点$v_{i}$邻接的点在图中的位置,链域指示下一条边或弧的结点,数据域存储和边和或弧相关的信息,如权值等。每个链表上附设有头指针。~\cite{严蔚敏2002数据结构}
\newline
\subsection{对基本运算的理解与实现}
总的来说,图的基本运算较简单,主要的难点在创建(\texttt{create})插入 (\texttt{insert})节点与删除 (\texttt{delete})节点和边。
因为需要对图的邻接链表进行改变,并且对相应的节点进行相应的插入和删除操作
同时要分配新的内存并创建新的节点。
如果操作不当,或者没有对用户的输入进行校验,插入或删除出错。
\subsection{单元测试}
通过编写完整的单元测试,将可能出现的错误都考虑清楚,尽量实现测试覆盖率达到100\%。
从而保证了程序在正确情况和极端情况下都能正常运行。

% ------------------------------------------------------------

\section{实验内容}\label{sec:test42}
    实验内容主要分为一下三个部分:
\begin{enumerate}
    \item 问题描述
    \item 系统设计
    \item 系统实现
\end{enumerate}
\section{问题描述}
\subsection{图的定义}
\begin{definition}\label{def:binarytree5}
    图 (\emph{Graph})中的数据原属通常称为顶点,$V$是顶点的有穷非空集合,$VR$是两个顶点之间关系的集合。若$(v,w)\in
    VR$则表示从$v$到$w$有一条弧,且称$v$为弧尾 (Tail),称$w$为弧头 (Head),此时的图称为有向图,若$VR$是对称的,则称此时的图为无向图。
\end{definition}
其中:
\begin{itemize}
    \item 数据节点可以为任意类型,但同一图中节点类型必须相同。
    \item 数据节点的个数$n$定义为图的大小,图里没有一个节点时称为空图。
    \item 具有$n(n-1)$条弧的图称为有向完全图。
    \item 有很少条边或弧的图称为稀疏图,反之则称为稠密图。
\end{itemize}

\subsection{图相关术语}

根据图中顶点间是否有路径,可以将图分为连通图与非连通图。

\begin{definition}
    在无向图中,如果从顶点$v$到顶点$v'$有路径,则称$v$和$v'$是连通的。如果对于图中任意两个顶点$v_i,\,v_j\in V,v_i,\,v_j$都是连通的,则称$G$是连通图。
\end{definition}

而由此我们也可以得到连通分量的概念。

\begin{definition}
    所谓连通分量,字的事无向图中的极大连通子图。
\end{definition}

有了连通分量的概念,我们可以得到生成树的概念。

\begin{definition}
    一个连通图的生成树是一个极小连通子图,她含有图中全部节点,但只有足以构成一棵树但$n-1$条边。
\end{definition}

邻接链表的存储特点是:只要确定了\emph{一个顶点},所有与之相连的顶点就可以很简单的通过遍历链表来获得。

\subsection{实验需要完成的基本操作}
\begin{enumerate}
\item 创建图:\texttt{CreateGraph (G)},\newline \textbf{\emph{初始条件}}是图G不存在 。\newline \textbf{\emph{操作结果}}是构造一个空的图。
\item 销毁图:\texttt{DestroyGraph (G)},\newline \textbf{\emph{初始条件}}是图G已存在。\newline \textbf{\emph{操作结果}}是销毁图G。
\item 清空图:\texttt{ClearGraph (G)},\newline \textbf{\emph{初始条件}}是图G已存在 。\newline \textbf{\emph{操作结果}}是将G重置为空图。
\item 查找节点:\texttt{LocateVex (G,e)},\newline \textbf{\emph{初始条件}}是图已存在 。\newline
    \textbf{\emph{操作结果}}返回查找到的结点指针,如无关键字为e的结点,返回NULL。
\item 节点赋值:\texttt{AssginVex (G,e,value)},\newline \textbf{\emph{初始条件}}是图G已存在 。\newline \textbf{\emph{操作结果}}是关键字为e的结点赋值为value。
\item 插入节点:\texttt{InsertVex (G,e,c)},\newline \textbf{\emph{初始条件}}是图G已存在且非空。\newline \textbf{\emph{操作结果}}插入结点c到G中,作为关键字为e的结点
\item 删除节点:\texttt{DeleteVex (G,i,e)},\newline \textbf{\emph{初始条件}}是图G已存在且非空。\newline
    \textbf{\emph{操作结果}}:是删除G中关键字为e的结点,同时删除所有与之相连的边。
\item 插入弧:\texttt{InsertArc  (G,src,dst)},\newline \textbf{\emph{初始条件}}是图G已存在且非空,且src dst均存在。\newline
    \textbf{\emph{操作结果}}从src插入一条到dst的弧。
\item 删除弧:\texttt{DeleteArc  (G,i,e)},\newline \textbf{\emph{初始条件}}是图G已存在且非空。\newline
    \textbf{\emph{操作结果}}:是删除G中关键字为e的弧,但保留弧所连接的结点;
\item 深度优先搜索:\texttt{DFS(G,visit ())},\newline \textbf{\emph{初始条件}}是图G已存在,\newline \textbf{\emph{操作结果}}根据深度优先对G的每个数据节点调用函数visit ()。
\item 广度优先搜索:\texttt{BFS(G,visit ())},\newline \textbf{\emph{初始条件}}是图G已存在,\newline \textbf{\emph{操作结果}}根据广度优先对G的每个数据节点调用函数visit ()。
\item 第一个邻接点:\texttt{FirstAdjVex(G,index)},\newline \textbf{\emph{初始条件}}是图G已存在,\newline
    \textbf{\emph{操作结果}}获得第index个结点的第一个邻接顶点。
\item 层级遍历:\texttt{NextAdjVex(G,Vindex,Aindex)},\newline \textbf{\emph{初始条件}}是图G已存在,\newline
    \textbf{\emph{操作结果}}获得第Vindex个顶点第Aindex+1个邻接顶点。
\end{enumerate}

\section{系统设计}
\subsection{总体设计}
本系统采用\emph{邻接链表}作为图的物理结构,实现有向图的基本运算。
\par
系统开始运行的时候默认不使用文件中的数据,但是用户随时可以将文件中的数据导入到内存中,同时提供数据保存的功能。
\subsection{有关常量和类型定义}
采取\texttt{C++}中的模版类来使图支持所有类型的数据,
\par
对于程序中可能出现的错误,进行统一规定:
\begin{enumerate}
    \item 对于用户输入不正确导致的在错误的位置插入或者删除,统一抛出\texttt{range\_error}。
    \item 对于其他可能发生的错误,统一抛出\texttt{runtime\_error}。
\end{enumerate}
\subsubsection{函数设计}
头文件中的函数原型声明见\autoref{appendix:h4}
函数时间复杂度分析见\autoref{tab:timeandspace4}
\begin{table}[ht]
\centering
\caption{函数时间与空间复杂度分析}
\label{tab:timeandspace4}
\begin{tabular}{@{}ccccc@{}}
\toprule
编号                          & 名称  & 函数签名 & 时间复杂度 & 空间复杂度 \\ \toprule
    \multicolumn{1}{c|}{1}  & 创建图 & \texttt{ createGraph()} & $O(V+E)$ &  $O(V+E)$ \\
    \multicolumn{1}{c|}{2}  & 销毁图& \texttt{ destroyGraph()} & $O(V+E)$ &  $O(V+E)$ \\
    \multicolumn{1}{c|}{3}  & 清空图& \texttt{ clearGraph()} & $O(1)$ &  $O(1)$   \\
    \multicolumn{1}{c|}{4}  & 获得结点值& \texttt{ getVexValue()} & $O(V)$ &  $O(1)$     \\
    \multicolumn{1}{c|}{5}  & 第一个邻接点& \texttt{ firstAdjVex()} & $O(V)$ &  $O(1)$     \\
    \multicolumn{1}{c|}{6}  & 下一个邻接点& \texttt{ nextAdjVex()} & $O(V+E)$ &  $O(V+E)$      \\
    \multicolumn{1}{c|}{7}  & 查找节点 & \texttt{ locate()} & $O(V)$ &  $O(1)$   \\
    \multicolumn{1}{c|}{8}  & 插入节点 & \texttt{ insertVex()} & $O(V)$ &  $O(1)$    \\
    \multicolumn{1}{c|}{9}  & 插入弧 & \texttt{ insertArc()} & $O(V+E)$ &  $O(1)$     \\
    \multicolumn{1}{c|}{10}  & 删除节点 & \texttt{ deleteVex()} & $O(V+E)$ &  $O(1)$  \\
    \multicolumn{1}{c|}{11}  & 删除节点 & \texttt{ deleteArc()} & $O(V+E)$ &  $O(1)$  \\
    \multicolumn{1}{c|}{12}  & 深度优先搜索& \texttt{ bfs()}   & $O(V+E)$ & $O(V+E)$ \\
    \multicolumn{1}{c|}{13}  & 广度优先搜索& \texttt{ bfs()}   & $O(V+E)$ & $O(V+E)$ \\ \bottomrule
\end{tabular}
\end{table}
\subsubsection{算法设计}
\paragraph{创建算法}如\autoref{alg:create4}所示,时间复杂度为$O(V+E)$,空间复杂度为$O(1)$
\par
\begin{algorithm}[H]
    \SetAlgoLined
    \KwIn{Def}
    \KwOut{Graph}
    \If{Def is invalid}{throw error}
    let $G$ be an empty graph\;
    \For{v=firstVex of Def; v\leq lastVex of Def; v++}{
        G.insertVex(v)\;
    }
    \For{a=firstArc of Def; a\leq lastArc of Def; a++}{
        G.insertArc(a)\;
    }
    \caption{Create}\label{alg:create4}
\end{algorithm}
\paragraph{插入算法}如\autoref{alg:insert4}述,其时间复杂度为$O(V)$,空间复杂度为$O(1)$
\par
\begin{algorithm}[H]
    \SetAlgoLined
    \KwIn{elem}
    \KwOut{none}
    \If{index out of range}{ throw error}
    let v = a new Vex with value elem\;
    G.InsertVex(v)\;
\caption{InsertVex}\label{alg:insert4}
\end{algorithm}
\paragraph{删除结点算法}如\autoref{alg:delete44}所述,其时间复杂度为$O(V+E)$,空间复杂度为$O(1)$
\begin{algorithm}[th]
    \SetAlgoLined
    \KwIn{value}
    \KwOut{none}
    \For{v=first vex of G; v < G.VexNumber; v++}{
        \For{a=first arc of v; v<v.ArcNumber;a++}{
            \If{a.to == value}{delete a}
        }
        \If{G.at(v)==value}{delete v}
    }
\caption{DeleteVex}\label{alg:delete44}
\end{algorithm}
\paragraph{插入弧算法}如\autoref{alg:insertArc}所示,时间复杂度为$O(V+E)$,空间复杂度为$O(1)$
\par
\begin{algorithm}[H]
    \SetAlgoLined
    \KwIn{from, to, weight}
    \KwOut{none}
    \For{v=first vex of G; v< G.VexNumber; v++}{
        \If{v==to}{
            v.insertArc(to,weight)\;
        }
    }
    \caption{InsertArc}\label{alg:insertArc}
\end{algorithm}
\paragraph{删除弧算法}如\autoref{alg:delete4}所示,时间复杂度为$O(V+E)$,空间复杂度为$O(1)$
\par
\begin{algorithm}[H]
    \SetAlgoLined
    \KwIn{from,to}
    \KwOut{none}
    \For{v=first vex of G; v< G.VexNumber; v++}{
        \If{v==from}{
            \For{a=v.first arc; a<v.ArcNumber;a++}{
                \If{a==to}{
                    delete a\;
                }
            }
        }
    }
    \caption{DeleteArc}\label{alg:delete4}
\end{algorithm}
\section{系统实现}
\subsection{开发环境}
本次实验中使用的环境如下:
\begin{enumerate}
    \item 操作系统版本 Darwin X86\_64 Kernel Version 19.2.0
    \item 编译器及其版本 clang++ version 11.0.0 (Apple LLVM version 11.0.0)
    \item 自动编译工具 CMake version 3.16.1
    \item 编程环境 NeoVim
\end{enumerate}
\subsection{代码结构及源代码}
本次实验采取了模块化的编码方式,
\footnote{具体代码结构见\autoref{appendix:structure}}
将图的不同部分的功能放置在不同源文件中,
\footnote{源代码见\autoref{appendix:lab4},测试代码见\autoref{appendix:test4}}
分别编译后链接,提高了代码的可维护性和编译速度,同时也更加容易编写单元测试,保证代码的正确性。
\subsection{代码亮点}
\begin{enumerate}
        \item 所有的代码均采用\emph{Google
            C/C++}标准代码规范以及\emph{ISOC++}委员会所推荐的\emph{CPP Core Guidelines}代码规范,同时使用\texttt{clang-format}和\texttt{clang-tidy}对代码进行格式化和规范化,符合现代化C++的规范。
        \item 所有代码均尽可能采用更新的C++标准,以提高可维护性以及可读性。
        \item 采用智能指针\footnote{需要C++11及以上的编译器支持}(\texttt{unique\_ptr})进行资源管理,避免了手动管理内存所可能造成的内存泄露等问题。
        \item 所有函数及变量均采用\texttt{auto}关键字进行声明,避免了错误的类型声明以及隐式的类型转换。
        \item 采用C++的模板类(\texttt{template})来实现图,体现出了图中能存放任何类型的节点、但同一图中只能存放单一类型节点的特点。
        \item 采用\texttt{CMake}自动生成\texttt{Makefile},从而简化了构建过程,同时能够实现跨平台编译。
        \item 使用现代化的C++测试框架\texttt{Catch2},简化了测试流程。
        \item 采用TDD\footnote{Test Driven Development}的开发方式,在保证效率的情况下尽可能地提高测试覆盖度,使系统更健壮,更容易维护。
        \item 使用\texttt{Sanitize}与\texttt{Valgrind}对代码进行检测,发现可能存在的内存泄漏或未定义行为。
        \item 使用\texttt{gcov, lcov, llvm-cov gcov}等工具对单元测试的覆盖度进行检测,并生成相应的测试报告。
\end{enumerate}

\subsection{系统测试}
本系统使用了\textit{Catch2}\footnote{url: https://github.com/catchorg/Catch2}作为测试框架,
对所有源代码编写了\textbf{完善}的单元测试,可以做到所有边界情况和越界情况以及正常情况全部覆盖,
同时程序中的每一个分支也都进行测试,使得测试覆盖度基本达到100\%。
% Please add the following required packages to your document preamble:
\subsubsection{构造函数测试}
\textbf{输入:}\texttt{Def}变量,创建的图定义。
\par
\textbf{输出:}\texttt{std::vector<T> res},图深度优先搜索的结果。
\par
\textbf{预计结果:}深度优先搜索正确。
\begin{table}[h]
\centering
\caption{构造函数测试}
\begin{tabular}{@{}ccccc@{}}
\toprule
\multicolumn{1}{c}{测试类型}    & \multicolumn{1}{c}{测试输入} & \multicolumn{1}{c}{理论输出} & \multicolumn{1}{c}{实际输出} & \multicolumn{1}{c}{测试结果} \\ \midrule
\multicolumn{1}{c|}{正确性测试}  & 随机一棵图的Def&深度优先搜索&深度优先搜索&正确\\
\multicolumn{1}{c|}{正确性测试}  & 空(默认构造函数)&1&1&正确\\
\multicolumn{1}{c|}{错误处理测试} & 错误定义& std::underflow\_error& std::underflow\_error& 正确\\ \bottomrule
\end{tabular}
\label{tab:inittest4}
\end{table}

\subsubsection{判定空图函数测试}
\textbf{输入:}\texttt{graph}变量,一个图。
\par
\textbf{输出:}\texttt{graph.empty()},图是否为空。
\par
\textbf{预计结果:}对于空图,输出\texttt{true},否则输出\texttt{false}。
\begin{table}[h]
\centering
    \caption{\texttt{判定空图函数测试}}
\begin{tabular}{@{}ccccc@{}}
\toprule
\multicolumn{1}{c}{测试类型}    & \multicolumn{1}{c}{测试输入} & \multicolumn{1}{c}{理论输出} & \multicolumn{1}{c}{实际输出} & \multicolumn{1}{c}{测试结果} \\ \midrule
\multicolumn{1}{c|}{正确性测试}  & 空graph&true&true&正确\\
\multicolumn{1}{c|}{正确性测试}  & 非空graph&false&false&正确\\ \bottomrule
\end{tabular}
\label{tab:emptytest4}
\end{table}


\subsubsection{销毁图测试}
\textbf{输入:}无
\par
\textbf{输出:}无
\par
\textbf{预计结果:}图被销毁
\begin{table}[h]
\caption{销毁图测试}
\centering
\begin{tabular}{@{}ccccc@{}}
\toprule
\multicolumn{1}{c}{测试类型}    & \multicolumn{1}{c}{测试输入} & \multicolumn{1}{c}{理论输出} & \multicolumn{1}{c}{实际输出} & \multicolumn{1}{c}{测试结果} \\ \midrule
\multicolumn{1}{c|}{正确性测试}  & 无&图被销毁&图被销毁&正确\\ \bottomrule
\end{tabular}
\label{tab:destorytest4}
\end{table}


\subsubsection{清空图测试}
\textbf{输入:}无
\par
\textbf{输出:}无
\par
\textbf{预计结果:}图被清空
\begin{table}[h]
    \centering
    \caption{清空图测试}
    \begin{tabular}{@{}ccccc@{}}
        \toprule
        \multicolumn{1}{c}{测试类型}    & \multicolumn{1}{c}{测试输入} & \multicolumn{1}{c}{理论输出} & \multicolumn{1}{c}{实际输出} &
        \multicolumn{1}{c}{测试结果} \\ \midrule
        \multicolumn{1}{c|}{正确性测试}  &无 &图被清空&图被清空&正确\\ \bottomrule
    \end{tabular}
    \label{tab:cleartest4}
\end{table}


\subsubsection{求第一个邻接点测试}
\textbf{输入:}一个图,index
\par
\textbf{输出:}value,第一个邻接点的值
\par
\textbf{预计结果:}输出与实际值相等。
\begin{table}[h]
    \centering
    \caption{求第一个邻接点}
    \begin{tabular}{@{}ccccc@{}}
        \toprule
        \multicolumn{1}{c}{测试类型}    & \multicolumn{1}{c}{测试输入} & \multicolumn{1}{c}{理论输出} & \multicolumn{1}{c}{实际输出} &
        \multicolumn{1}{c}{测试结果} \\ \midrule
        \multicolumn{1}{c|}{正确性测试}  & 5&5&5&正确\\
        \multicolumn{1}{c|}{正确性测试}  & 空图&0&0&正确\\ \bottomrule
    \end{tabular}
    \label{tab:lengthtest4}
\end{table}


\subsubsection{查找节点测试}
\textbf{输入:}\texttt{key}变量,要获取节点的key。
\par
\textbf{输出:}\texttt{element},获得到的节点。
\par
\textbf{预计结果:}想要获取到的节点与实际获取的节点相等。
\begin{table}[h]
    \centering
    \caption{查找节点测试}
    \begin{tabular}{@{}ccccc@{}}
        \toprule
        \multicolumn{1}{c}{测试类型}    & \multicolumn{1}{c}{测试输入} & \multicolumn{1}{c}{理论输出} & \multicolumn{1}{c}{实际输出} &
        \multicolumn{1}{c}{测试结果} \\ \midrule
        \multicolumn{1}{c|}{正确性测试}  & 0&``aaa''&``aaa''&正确\\
        \multicolumn{1}{c|}{正确性测试}  & 1&``bbb''&``bbb''&正确\\
        \multicolumn{1}{c|}{正确性测试}  & 3&``ddd''&``ddd''&正确\\
        \multicolumn{1}{c|}{错误处理测试} & 10000& std::overflow\_error& std::overflow\_error& 正确\\ \bottomrule
    \end{tabular}
    \label{tab:gettest4}
\end{table}

\subsubsection{下一个邻接点测试}
\textbf{输入:}\texttt{Vindex, Aindex}变量,节点的索引和弧的索引。
\par
\textbf{输出:}\texttt{value},节点的值,没找到时返回-1。
\par
\textbf{预计结果:}与预期相等。
\begin{table}[h]
    \centering
    \caption{下一个邻接点测试}
    \begin{tabular}{@{}ccccc@{}}
        \toprule
        \multicolumn{1}{c}{测试类型}    & \multicolumn{1}{c}{测试输入} & \multicolumn{1}{c}{理论输出} & \multicolumn{1}{c}{实际输出} &
        \multicolumn{1}{c}{测试结果} \\ \midrule
        \multicolumn{1}{c|}{正确性测试}  & 1,``aaa''&0&0&正确\\
        \multicolumn{1}{c|}{正确性测试}  & 3,``bbb''&1&1&正确\\
        \multicolumn{1}{c|}{正确性测试}  & 1000,``magic''&-1&-1&正确\\ \bottomrule
    \end{tabular}
    \label{tab:locatetest4}
\end{table}


\subsubsection{删除结点测试}
\textbf{输入:}\texttt{e}变量,要删除的位置。
\par
\textbf{输出:}无
\par
\textbf{预计结果:}节点被删除。
\begin{table}[h]
    \centering
    \caption{删除结点测试}
    \begin{tabular}{@{}ccccc@{}}
        \toprule
        \multicolumn{1}{c}{测试类型}    & \multicolumn{1}{c}{测试输入} & \multicolumn{1}{c}{理论输出} & \multicolumn{1}{c}{实际输出} &
        \multicolumn{1}{c}{测试结果} \\ \midrule
        \multicolumn{1}{c|}{正确性测试}  & ``ddd''&``ccc''&``ccc''&正确\\
        \multicolumn{1}{c|}{错误处理测试} & ``aaa''& std::underflow\_error& std::underflow\_error& 正确\\ \bottomrule
    \end{tabular}
    \label{tab:priortest4}
\end{table}

\subsubsection{插入弧测试}
\textbf{输入:}\texttt{from,to,weight}变量,要插入的位置和权重。
\par
\textbf{输出:}无
\par
\textbf{预计结果:}弧正常插入。
\begin{table}[h]
    \centering
    \caption{插入弧测试}
    \begin{tabular}{@{}ccccc@{}}
        \toprule
        \multicolumn{1}{c}{测试类型}    & \multicolumn{1}{c}{测试输入} & \multicolumn{1}{c}{理论输出} & \multicolumn{1}{c}{实际输出} &
        \multicolumn{1}{c}{测试结果} \\ \midrule
        \multicolumn{1}{c|}{正确性测试}  & (1,2,999)& 正确插入 &正确插入 &正确\\
        \multicolumn{1}{c|}{错误处理测试} & (1000,2,999)& std::overflow\_error& std::overflow\_error& 正确\\ \bottomrule
    \end{tabular}
    \label{tab:nexttest4}
\end{table}

\subsubsection{插入节点测试}
\textbf{输入:}\texttt{(index, element, c)}变量,要插入的位置和要插入的节点。
\par
\textbf{输出:}\texttt{(element)},插入的节点。
\par
\textbf{预计结果:}两者对应相等。
\begin{table}[h]
    \centering
    \caption{插入节点测试}
    \begin{tabular}{@{}ccccc@{}}
        \toprule
        \multicolumn{1}{c}{测试类型}    & \multicolumn{1}{c}{测试输入} & \multicolumn{1}{c}{理论输出} & \multicolumn{1}{c}{实际输出} &
        \multicolumn{1}{c}{测试结果} \\ \midrule
        \multicolumn{1}{c|}{正确性测试}  & (1, ``sss '')&``sss''& ``sss''&正确\\
        \multicolumn{1}{c|}{正确性测试}  & (2, `` sss '')&``sss''& ``sss''&正确\\
        \multicolumn{1}{c|}{错误处理测试} & 0& std::underflow\_error& std::underflow\_error& 正确\\
        \multicolumn{1}{c|}{错误处理测试} & 100& std::overflow\_error& std::overflow\_error& 正确\\ \bottomrule
    \end{tabular}
    \label{tab:inserttest4}
\end{table}


\subsubsection{删除弧测试}
\textbf{输入:}\texttt{from,to}变量,要删除的位置。
\par
\textbf{输出:}无
\par
\textbf{预计结果:}弧被正确删除。
\begin{table}[h]
    \centering
    \caption{删除弧测试}
    \begin{tabular}{@{}ccccc@{}}
        \toprule
        \multicolumn{1}{c}{测试类型}    & \multicolumn{1}{c}{测试输入} & \multicolumn{1}{c}{理论输出} & \multicolumn{1}{c}{实际输出} &
        \multicolumn{1}{c}{测试结果} \\ \midrule
        \multicolumn{1}{c|}{正确性测试}  & (1,2)&正确删除&正确删除&正确\\
        \multicolumn{1}{c|}{错误处理测试} & 0& std::underflow\_error& std::underflow\_error& 正确\\
        \multicolumn{1}{c|}{错误处理测试} & 100& std::overflow\_error& std::overflow\_error& 正确\\ \bottomrule
    \end{tabular}
    \label{tab:deletetest4}
\end{table}


\subsubsection{深度优先搜索测试}
\textbf{输入:}无
\par
\textbf{输出:}深度优先搜索序列
\par
\textbf{预计结果:}与实际序列相等。
\begin{table}[h]
    \caption{深度优先搜索测试}
    \centering
    \begin{tabular}{@{}ccccc@{}}
        \toprule
        \multicolumn{1}{c}{测试类型}    & \multicolumn{1}{c}{测试输入} & \multicolumn{1}{c}{理论输出} & \multicolumn{1}{c}{实际输出} &
        \multicolumn{1}{c}{测试结果} \\ \midrule
        \multicolumn{1}{c|}{正确性测试}  & /&(1,2,4,5,3)&(1,2,4,5,3)&正确\\ \bottomrule
    \end{tabular}
    \label{tab:traversetest4}
\end{table}
\subsubsection{广度优先搜索测试}
\textbf{输入:}无
\par
\textbf{输出:}广度优先搜索序列
\par
\textbf{预计结果:}与实际序列相等。
\begin{table}[h]
    \caption{广度优先搜索测试}
    \centering
    \begin{tabular}{@{}ccccc@{}}
        \toprule
        \multicolumn{1}{c}{测试类型}    & \multicolumn{1}{c}{测试输入} & \multicolumn{1}{c}{理论输出} & \multicolumn{1}{c}{实际输出} &
        \multicolumn{1}{c}{测试结果} \\ \midrule
        \multicolumn{1}{c|}{正确性测试}  & /&(1,2,3,5,4)&(1,2,3,5,4)&正确\\ \bottomrule
    \end{tabular}
    \label{tab:bfs4}
\end{table}

\subsubsection{测试小结}
所有测试均通过,并且绝大部分的边界条件也都覆盖到了。可以认为程序中不存在逻辑错误,
了所有基本功能,并且能够执行文件操作,同时也可以在客户端进行多图操作。
\par
通过测试,证明了系统实现的完整性和正确性,确保了系统的良好运行。
\par
\autoref{fig:cov4}中为测试覆盖度\footnote{更具体的可以见\texttt{out}文件夹}

\begin{figure}
\centering
\caption{测试覆盖度}\label{fig:cov4}
\includegraphics[scale=.5]{cov.png}
\end{figure}

\section{实验结果与分析}\label{sec:test44}
本次实验加深了对图的概念、基本运算的理解,
掌握了图的基本运算的实现。
深刻理解了图的\emph{逻辑结构}和\emph{物理结构}之间的关系。
\par 整个实验均在\texttt{UNIX}环境下编程
所有的代码均采用\emph{Google C/C++}标准代码规范,
通过\texttt{clang-format}和\texttt{clang-tidy}进行格式化和规范化。
\par
同时,在编码过程中,我尽可能地使用了更加现代化的C++代码,例如使用智能指针(\textit{unique\_ptr})来进行资源管理,从而避免了手动管理内存而可能带来的内存泄漏等问题、
例如使用\texttt{auto}关键字来声明函数与变量,从而减少了错误的类型声明或不正确的隐式类型转换。
使得代码可读性更高,更容易维护,更加健壮。这样的规范和编码习惯有助于以后在工作中更高效地完成工作任务。
\par
本系统完整的实现了课程要求的全部功能,并且实现了多图管理和文件存储功能,
系统健壮性良好,可以应对各种情况的输入,且能输出相应的错误提示。
系统测试覆盖度接近100\%,可以认为不会发生逻辑错误
\par


\backmatter

\begin{ack}
    感谢老师、助教和同学在本次实验中的对我的帮助,特别是许老师和助教许学长,在实验中为我提供了很多建议。
    \par
    在此衷心地对他们表示感谢!
    \par
    特别感谢教材\cite{严蔚敏2002数据结构}与习题册\cite{严蔚敏1998数据结构题集}对我的帮助
\end{ack}

\bibliography{book}

\appendix

\chapter{实验源代码}\label{appendix:1}
\section{文件结构及代码量统计}\label{appendix:structure}
\subsection{文件结构}
\lstinputlisting{../tree.dat}
\subsection{代码量统计}
\lstinputlisting{../code.dat}
\section{实验一源代码}\label{appendix:lab1}
\subsection{头文件定义}\label{appendix:h1}
\emph{List.hh}
\lstinputlisting[language=c++]{../include/lab/1/List.hh}
\subsection{线性表实现代码}
\emph{构造函数与初始化——List.cc}
\lstinputlisting[language=c++]{../src/lab/1/List.cc}
\emph{元素操作——elem.cc}
\lstinputlisting[language=c++]{../src/lab/1/elem.cc}
\emph{插入删除等操作——modify.cc}
\lstinputlisting[language=c++]{../src/lab/1/modify.cc}
\emph{文件操作——fs.cc}
\lstinputlisting[language=c++]{../src/lab/1/fs.cc}
\emph{CMake配置}
\lstinputlisting{../src/lab/1/CMakeLists.txt}
\subsection{测试代码}\label{appendix:test1}
\emph{构造函数与初始化测试——List\_test.cc}
\lstinputlisting[language=c++]{../test/lab/1/List_test.cc}
\emph{元素操作测试——elem\_test.cc}
\lstinputlisting[language=c++]{../test/lab/1/elem_test.cc}
\emph{插入删除等操作测试——modify\_test.cc}
\lstinputlisting[language=c++]{../test/lab/1/modify_test.cc}
\emph{文件操作测试——fs\_test.cc}
\lstinputlisting[language=c++]{../test/lab/1/fs_test.cc}

\section{实验二源代码}\label{appendix:lab2}
\subsection{头文件定义}\label{appendix:h2}
\emph{List.hh}
\lstinputlisting[language=c++]{../include/lab/2/List.hh}
\emph{Node.hh}
\lstinputlisting[language=c++]{../include/lab/2/Node.hh}
\subsection{线性表实现代码}
\emph{节点构造函数与初始化——Node.cc}
\lstinputlisting[language=c++]{../src/lab/2/Node.cc}
\emph{线性表构造函数与初始化——List.cc}
\lstinputlisting[language=c++]{../src/lab/2/List.cc}
\emph{元素操作——elem.cc}
\lstinputlisting[language=c++]{../src/lab/2/elem.cc}
\emph{插入删除等操作——modify.cc}
\lstinputlisting[language=c++]{../src/lab/2/modify.cc}
\emph{CMake配置}
\lstinputlisting{../src/lab/2/CMakeLists.txt}
\subsection{测试代码}\label{appendix:test2}
\emph{节点构造函数与初始化测试——Node\_test.cc}
\lstinputlisting[language=c++]{../test/lab/2/Node_test.cc}
\emph{线性表构造函数与初始化测试——List\_test.cc}
\lstinputlisting[language=c++]{../test/lab/2/List_test.cc}

\section{实验三源代码}\label{appendix:lab3}
\subsection{头文件定义}\label{appendix:h3}
\emph{Tree.hh}
\lstinputlisting[language=c++]{../include/lab/3/Tree.hh}
\emph{Node.hh}
\lstinputlisting[language=c++]{../include/lab/3/Node.hh}
\subsection{二叉树实现代码}
\emph{节点构造函数与初始化——Node.cc}
\lstinputlisting[language=c++]{../src/lab/3/Node.cc}
\emph{树的构造与操作——modify.cc}
\lstinputlisting[language=c++]{../src/lab/3/Tree.cc}
\emph{CMake配置}
\lstinputlisting{../src/lab/3/CMakeLists.txt}
\subsection{测试代码}\label{appendix:test3}
\emph{节点构造函数与初始化测试——Node\_test.cc}
\lstinputlisting[language=c++]{../test/lab/3/Node_test.cc}
\emph{构造函数与初始化测试——Tree\_test.cc}
\lstinputlisting[language=c++]{../test/lab/3/Tree_test.cc}

\section{实验四源代码}\label{appendix:lab4}
\subsection{头文件定义}\label{appendix:h4}
\emph{Graph.hh}
\lstinputlisting[language=c++]{../include/lab/4/Graph.hh}
\subsection{图实现代码}
\emph{节点构造函数与初始化——Node.cc}
\lstinputlisting[language=c++]{../src/lab/4/Node.cc}
\emph{树的构造与操作——modify.cc}
\lstinputlisting[language=c++]{../src/lab/4/Graph.cc}
\emph{CMake配置}
\lstinputlisting{../src/lab/4/CMakeLists.txt}
\subsection{测试代码}\label{appendix:test4}
\emph{图功能测试——Graph\_test.cc}
\lstinputlisting[language=c++]{../test/lab/4/Graph_test.cc}

\end{document}
\endinput
%%
